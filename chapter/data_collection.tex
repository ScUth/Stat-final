\chapter*{Data Collection}
We collected the data with a survey using Google Forms. Our target is people in different age groups. The survey is mostly in Thai and was accepting the results from April 12, 2025, until April 17, 2025, and 57 people responded to the survey.\\*[5pt]
The survey questions are : 
\begin{multicols}{2}
    \begin{enumerate}
        \item What is your age?
        \item What is your gender?
        \item What is your weight (kg.) ?
        \item What is your heart rate while answering the survey?
        \item What is your approximate average sleeping duration (Hr.min) ?
        \item How much screen duration did you spend on Sunday (Hr.min) ?
        \item How much screen duration did you spend on Monday (Hr.min) ?
        \item How much screen duration did you spend on Tuesday (Hr.min) ?
        \item How much screen duration did you spend on Wednesday (Hr.min) ?
        \item How much screen duration did you spend on Thursday (Hr.min) ?
        \item How much screen duration did you spend on Friday (Hr.min) ?
        \item How much screen duration did you spend on Saturday (Hr.min) ?
        \item What is your most used application type?
        \item How much notification did you receive on Sunday?
        \item How much notification did you receive on Monday?
        \item How much notification did you receive on Tuesday?
        \item How much notification did you receive on Wednesday?
        \item How much notification did you receive on Thursday?
        \item How much notification did you receive on Friday?
        \item How much notification did you receive on Saturday?
        \item What is the current condition of your left eye?
        \begin{enumerate}[label=21.\arabic*]
            \item How is your eyesight (normal, near-sighted, far-sighted or Myopic Presbyopia) ?
            \begin{enumerate}[label=21.1.\arabic*]
                \item If you are near-sighted, how much?
                \item If you are far-sighted, how much?
                \item If you have compound vision, how much?
            \end{enumerate}
            \item Other abnormalities?
        \end{enumerate}
        \item What is the current condition of your right eye (normal, near-sighted, far-sighted or compound vision) ?
        \begin{enumerate}[label=22.\arabic*]
            \item How is your eyesight (normal, near-sighted, far-sighted or Myopic Presbyopia) ?
            \begin{enumerate}[label=22.1.\arabic*]
                \item If you are near-sighted, how much?
                \item If you are far-sighted, how much?
                \item If you have compound vision, how much?
            \end{enumerate}
            \item Other abnormalities?
        \end{enumerate}
    \end{enumerate}
\end{multicols}
\newpage
\noindent
After we collected the data, we cleaned up the data. We found that there was data that had the same answers in every question(both surveys have the same answer and the answers came from the example provided in the form) except gender and age, so we did not use their answers. Another problem is the respondents did not follow the instructions that we provided, so the answers might not be accurate.\\*[5pt]
We found that most of the respondents were Male with 52.2\%, followed by female with 47.8\%. The majority of the respondents are 18 \& 19 years old (45.65\%), followed by older ages.\\*[5pt]
If we grouped the data by type of application used the most per day, 58.7\% of the respondents spent the most time on entertainment applications, followed by social media at 34.8\%, education at 4.3\%, and information \& reading at 2.2\%. \\*[5pt]
If we grouped the data by day of the week that the respondents spent their time on the screen the most, 23.9\% of the respondents spent time on their screen on Sunday the most, followed by Thursday \& Tuesday, Wednesday \& Friday \& Saturday and Monday with 15.3\%, 13.0\% and 6.5\% of the respondents respectively.\\*[5pt]
If we grouped the data by day of the week that the respondents spent their time on the screen the least, 28.3\% of the respondents spent time on their screen on Monday the least, followed by Thursday \& Saturday, Wednesday \& Sunday, Tuesday and Friday with 19.6\%, 10.9\%, 8.7\% and 2.2\% of the respondents respectively.\\*[5pt]