\chapter*{Conclusion}
From the question that we asked the respondents, we want to evaluate how demographic factors (age, weight) influence screen behavior and to analyze the impact of screen duration on daily routines, including which days of the week have the highest and lowest usage.\\ \par
We found that age has a weak negative relationship with screen duration, with a correlation coefficient of -0.3340. Weight does not have any relationship with screen duration, with a correlation coefficient of -0.0137. The median of screen duration is 6.071, the median of weight is 59.95 and the median age is 21.\\ \par
The day that the most respondents have spent their screen duration is Sunday with 23.9\% and the lowest screen duration is Monday with 28.3\%. The median of sleep duration is 7.333. However, we found that screen duration and sleep duration are not correlated with the correlation coefficient of -0.1431.\\\par
The flaw of this is that the majority of people who answer the survey are around 18 & 19 years old which might cause the result to be inaccurate.
