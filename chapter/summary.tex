\chapter*{Summary}
In today's world, mobile phones have become an important part of our daily lives. People use them for communication, entertainment, studying, and even working. Because of this, many people spend a lot of time looking at their phone screens every day. This is called ``screen time.''\\ \par
With the rise of smartphones in daily routines, we aim to understand how much time people spend on their smartphones and how this may relate to different personal conditions, such as weight, heart rate, age, gender, eyesight, and sleep duration. In addition, we explore which days of the week people tend to have the highest and lowest screen time. We also look into which types of applications are used the most, such as social media, entertainment, or productivity apps. This information can help us understand the impact of smartphone usage on health and daily habits.\\ \par
Following our analysis, we found that age has a weak negative correlation with screen time, meaning that older people tend to spend slightly less time on their smartphones. In contrast, weight and sleep duration showed no correlation with screen time. Furthermore, the data indicated that respondents spent the most time on their mobile phones on Sundays, while Mondays recorded the lowest usage.

\begin{flushright}
    \begin{tabular}{cclll}
        \multicolumn{2}{c}{Group 6}     &  &  &  \\*[0.2cm]
        6710545873 & Vorapop Prasertkul &  &  &  \\
        6710545521 & Chaiyapat Kumtho   &  &  &  \\
        6710545741 & Pasin Tongtip      &  &  &  \\
        6710545989 & Amornrit Sirikham  &  &  & 
    \end{tabular}
\end{flushright}