\documentclass[a4paper, 5pt]{report}
%\documentclass{standalone}

\usepackage{tocloft}
\usepackage{xltxtra} 

%\XeTeXlinebreaklocale "th"
%\XeTeXlinebreakskip = 0pt plus 1pt   %
\usepackage[fontsize=11pt]{fontsize}
\usepackage[T1]{fontenc}
\usepackage{lmodern}
%layout
\setlength{\voffset}{0.125cm}
%\setlength{\hoffset}{1.5pt}
\oddsidemargin = 1cm
%\marginparwidth = -3pt
\headsep = 10pt
\textwidth = 435pt
\textheight = 720pt
\topmargin = -23pt
\usepackage[top=1in, pass]{geometry}
%layout
% \setmainfont{AngsanaUPC}
\usepackage{calligra}
\usepackage{iflang} 
\usepackage{tikz}
\usepackage{fontspec}
\usepackage{polyglossia}
\usepackage{layout}
\usepackage{floatrow}
\usepackage{filecontents}
\usepackage{amsmath}
\usepackage[options]{stackengine}
\usepackage{pgfplots}
\usepackage[options]{cancel}
\pgfplotsset{compat=1.18}
\usepackage{subcaption}
\usetikzlibrary{plotmarks}
\usetikzlibrary{calc}
\usepgfplotslibrary{fillbetween}
\usetikzlibrary{spy}
\usetikzlibrary{decorations,calligraphy}
\tikzexternalize
\usepackage{etoolbox}
\usepackage{letltxmacro}
\usepackage{xfrac}
\usepackage[dvipsnames]{xcolor}
\usepackage{colortbl}
\usepackage[options]{amssymb}
\usepackage{calc}
\usepackage[tbtags]{amsmath}
\usepackage[options]{multicol}
\usepackage[options]{mathtools}
\usepackage{amsfonts}
\usepackage[options]{multirow}
\usepgfplotslibrary{statistics}
\usepackage{siunitx}
\usepackage{graphicx}
\usepackage{longtable}
\usepackage{array}
\usepackage[options]{lipsum}
\usepackage{slashbox}
\usepackage{adjustbox}
\usepackage{hhline}

\def\dotfill#1{\cleaders\hbox to #1{.}\hfill}
\makeatletter
\def\myrulefill{\leavevmode\leaders\hrule height .8ex width 1ex depth -0.6ex\hfill\kern\z@}
\makeatother

\newcommand{\Z}{\mathbb{Z}}
\newcommand{\scb}{\scalebox}
\newcommand\xof[3][1ex]{\ensurestackMath{%
  \setbox0=\hbox{$#2$}%
  \setbox2=\hbox{$#3$}%
  \kern\wd0\kern-\dimexpr#1\relax%
  \stackinset{r}{#1}{b}{\dimexpr#1-1ex}{\mathrlap{#3}}{%
    \stackinset{l}{#1}{t}{\dimexpr#1-1ex}{\mathllap{#2}}{\big/}%
  }%
  \kern-\dimexpr#1\relax\kern\wd2%
}}

\makeatletter
% \overarrow@ and \arrowfill@ are defined in amsmath
\newcommand*{\overrightharpoonup}{\mathpalette{\overarrow@\rightharpoonupfill@}}
\newcommand*{\rightharpoonupfill@}{\arrowfill@\relbar\relbar\rightharpoonup}
\makeatother

\DeclareFontFamily{U}{matha}{\hyphenchar\font45}
\DeclareFontShape{U}{matha}{m}{n}{
      <5> <6> <7> <8> <9> <10> gen * matha
      <10.95> matha10 <12> <14.4> <17.28> <20.74> <24.88> matha12
      }{}
\DeclareSymbolFont{matha}{U}{matha}{m}{n}
\DeclareMathSymbol{\varrightharpoonup}{3}{matha}{"E1}

\newcommand\Cancel[2][black]{\renewcturnommand\CancelColor{\color{#1}}\cancel{#2}}

\newcommand{\wideunderline}[2][2em]{%
  \underline{\makebox[\ifdim\width>#1\width\else#1\fi]{#2}}%
}

\newcommand*{\permcomb}[4][0mu]{{{}^{#3}\mkern#1#2_{#4}}}
\newcommand*{\perm}[1][-3mu]{\permcomb[#1]{P}}
\newcommand*{\comb}[1][-1mu]{\permcomb[#1]{C}}

\newcommand\checkmark[1][]{%
  \tikz[scale=0.4,#1]{\fill(0,.35) -- (.25,0) -- (1,.7) -- (.25,.15) -- cycle;}%
}
\newcommand\crossmark[1][]{%
  \tikz[scale=0.4,#1]{
    \fill(0,0)--(0.1,0) .. controls (0.5,0.4) .. (1,0.7)--(0.9,0.7) ..  controls (0.5,0.5) ..(0,0.1) --cycle;
    \fill(1,0.1)--(0.9,0.1) .. controls (0.5,0.3) .. (0,0.7)--(0.1,0.7) .. controls (0.5,0.4) ..(1,0.2) --cycle;
  }%
}

\usetikzlibrary{shapes.misc,shadows}

\DeclareMathOperator{\asec}{arcsec}
\DeclareMathOperator{\asin}{arcsin}
\DeclareMathOperator{\acot}{arccot}
\DeclareMathOperator{\atan}{arctan}
\DeclareMathOperator{\acsc}{arccsc}
\DeclareMathOperator{\sech}{sech}
\DeclareMathOperator{\csch}{csch}

\newcommand\tab[1][1cm]{\hspace*{#1}}
\renewcommand*\arraystretch{1.3}
%roman num
\newcommand{\RomanI}{I}
\newcommand{\joinR}{\hspace{-.1em}}
\newcommand{\RomanII}{\mbox{\RomanI\joinR\RomanI}}
\newcommand{\RomanIII}{\mbox{\RomanI\joinR\RomanII}}
\newcommand{\RomanIV}{\mbox{\RomanI\joinR\RomanV}}
\newcommand{\RomanV}{V}
\newcommand{\RomanVI}{\mbox{\RomanV\joinR\RomanI}}
\newcommand{\RomanVII}{\mbox{\RomanV\joinR\RomanII}}
\newcommand{\RomanVIII}{\mbox{\RomanV\joinR\RomanIII}}
\newcommand{\RomanIX}{\mbox{\RomanI\joinR\RomanX}}
\newcommand{\RomanX}{X}
%roman num

%for textcolor
\newcommand{\tcr}{\textcolor{red}}
\newcommand{\tcb}{\textcolor{blue}}
\newcommand{\tcp}{\textcolor{Plum}}
\newcommand{\tct}{\textcolor{teal}}
\newcommand{\tcm}{\textcolor{magenta}}
\newcommand{\tco}{\textcolor{orange}}
\newcommand{\dil}{\displaystyle\int\limits}
\newcommand{\il}{\int\limits}
\newcommand{\iil}{\iint\limits}
\newcommand{\lln}{\lim\limits_{n\to\infty}}
\newcommand{\sn}{\sum\limits^\infty_{n=1}}
\newcommand{\dsn}{\displaystyle\sum\limits^\infty_{n=1}}
\newcommand{\dso}{\displaystyle\sum\limits^\infty_{n=0}}
\newcommand{\dsl}{\displaystyle\sum\limits}
\newcommand{\dsi}{\displaystyle\sum\limits^\infty}
\newcommand{\eqdef}{\overset{\mathrm{def}}{=\joinrel=}}
\newcommand{\stl}{\bigg|}
%for textcolor
\usepackage{fancyhdr}
\usepackage{pdfpages}

% \pagenumbering{gobble}

\makeatletter
\def\equalsfill{$\m@th\mathord=\mkern-7mu
\cleaders\hbox{$\!\mathord=\!$}\hfill
\mkern-7mu\mathord=$}
\makeatother
\usepackage[autostyle=false, style=english]{csquotes}
\MakeOuterQuote{"}
\usepackage[shortlabels]{enumitem}
\setlist[enumerate]{font=\color{Red}}
\usepackage{listings}
\usepackage[table,xcdraw]{xcolor}

\begin{document}
    \chapter*{Summary}
Mobile phones have become an important part of our daily lives. People use them for communication, entertainment, studying, and even working. Because of this, many people spend a lot of time looking at their phone screens every day. This is called ``screen time''.\\*[5pt]
With the rise of smartphones in daily routines, we aim to understand how much time people spend on their smartphones and how this may relate to different personal conditions, such as weight, heart rate, age, gender, eyesight, and sleep duration. In addition, we explore which days of the week people tend to have the highest and lowest screen time. We also look into which application types are used the most, such as social media, entertainment, or productivity apps. This information can help us understand the impact of smartphone usage on health and daily habits.\\*[5pt]
Following our analysis, we found that age has a weak negative correlation with screen time, meaning that older people tend to spend slightly less time on their smartphones. In contrast, weight and sleep duration showed no correlation with screen time. Furthermore, the data indicated that respondents spent the most time on their mobile phones on Sundays, while Mondays recorded the lowest usage.\\*[5pt]

\begin{flushright}
    \begin{tabular}{cclll}
        \multicolumn{2}{c}{Group 6}     &  &  &  \\*[0.2cm]
        6710545873 & Vorapop Prasertkul &  &  &  \\
        6710545521 & Chaiyapat Kumtho   &  &  &  \\
        6710545741 & Pasin Tongtip      &  &  &  \\
        6710545989 & Amornrit Sirikham  &  &  & 
    \end{tabular}
\end{flushright}
    \clearpage
    \chapter*{Data Collection}
We collected the data with a survey using Google Forms. Our target is people in different age groups. The survey is mostly in Thai and was accepting the result from April 12, 2025 until April 17, 2025, and 57 people responded to the survey.\\*[5pt]
The survey questions are : 
\begin{multicols}{2}
    \begin{enumerate}
        \item What is your age?
        \item What is your gender?
        \item What is your weight (kg.) ?
        \item What is your heart rate while answering the survey?
        \item What is your approximate average sleeping duration (Hr.min) ?
        \item How much screen duration did you spend on Sunday (Hr.min) ?
        \item How much screen duration did you spend on Monday (Hr.min) ?
        \item How much screen duration did you spend on Tuesday (Hr.min) ?
        \item How much screen duration did you spend on Wednesday (Hr.min) ?
        \item How much screen duration did you spend on Thursday (Hr.min) ?
        \item How much screen duration did you spend on Friday (Hr.min) ?
        \item How much screen duration did you spend on Saturday (Hr.min) ?
        \item What is your most used application type?
        \item How much notification did you receive on Sunday?
        \item How much notification did you receive on Monday?
        \item How much notification did you receive on Tuesday?
        \item How much notification did you receive on Wednesday?
        \item How much notification did you receive on Thursday?
        \item How much notification did you receive on Friday?
        \item How much notification did you receive on Saturday?
        \item What is the current condition of your left eye?
        \begin{enumerate}[label=21.\arabic*]
            \item How is your eyesight (normal, near-sighted, far-sighted or Myopic Presbyopia) ?
            \begin{enumerate}[label=21.1.\arabic*]
                \item If you are near-sighted, how much?
                \item If you are far-sighted, how much?
                \item If you have compound vision, how much?
            \end{enumerate}
            \item Other abnormalities?
        \end{enumerate}
        \item What is the current condition of your right eye (normal, near-sighted, far-sighted or compound vision) ?
        \begin{enumerate}[label=22.\arabic*]
            \item How is your eyesight (normal, near-sighted, far-sighted or Myopic Presbyopia) ?
            \begin{enumerate}[label=22.1.\arabic*]
                \item If you are near-sighted, how much?
                \item If you are far-sighted, how much?
                \item If you have compound vision, how much?
            \end{enumerate}
            \item Other abnormalities?
        \end{enumerate}
    \end{enumerate}
\end{multicols}
\newpage
After we collected the data, we cleaned up the data. We found that there was data that had the same answers in every question(both surveys have the same answer and the answers came from the example provided in the form) except gender and age, so we did not use their answers. Another problem is the respondents did not follow the instructions that we provided, so the answers might not be accurate.\\ \par
We found that most of the respondents were Male with 52.2\%, followed by female with 47.8\%. Majority of the respondents are 18 & 19 years old (45.65\%), followed by older ages.\\ \par
If we grouped the data by type of application used the most per day, 58.7\% of the respondents spent the most time on entertainment application, followed by social media with 34.8\%, education with 4.3\% and information & reading with 2.2\%.\\ \par
If we grouped the data by day of week that the respondents spend their time on the screen the most, 23.9\% of the respondents spent time on their screen on Sunday the most, followed by Thursday & Tuesday, Wednesday & Friday & Saturday and Monday with 15.3\%, 13.0\% and 6.5\% of the respondents respectively.\\ \par
If we grouped the data by day of week that the respondents spend their time on the screen the least, 28.3\% of the respondents spent time on their screen on Monday the least, followed by Thursday & Saturday, Wednesday & Sunday, Tuesday and Friday with 19.6\%, 10.9\%, 8.7\% and 2.2\% of the respondents respectively.\\ \par

    \clearpage
    \chapter*{Methodology}
\section*{Probability}
\subsection*{Screen time vs Age}
\begin{table}[hb!]
    \begin{adjustbox}{width = \textwidth, center}
        \begin{tabular}{|c|r|r|r|r|r|r|r|r|r|r|r|r|r|r|}
        \hline
                                                                    & \multicolumn{1}{c|}{\cellcolor[HTML]{F4CCCC}2} & \multicolumn{1}{c|}{\cellcolor[HTML]{F4CCCC}3} & \multicolumn{1}{c|}{\cellcolor[HTML]{F4CCCC}4} & \multicolumn{1}{c|}{\cellcolor[HTML]{F4CCCC}5} & \multicolumn{1}{c|}{\cellcolor[HTML]{F4CCCC}6} & \multicolumn{1}{c|}{\cellcolor[HTML]{F4CCCC}7} & \multicolumn{1}{c|}{\cellcolor[HTML]{F4CCCC}8} & \multicolumn{1}{c|}{\cellcolor[HTML]{F4CCCC}9}  & \multicolumn{1}{c|}{\cellcolor[HTML]{F4CCCC}10} & \multicolumn{1}{c|}{\cellcolor[HTML]{F4CCCC}11} & \multicolumn{1}{c|}{\cellcolor[HTML]{F4CCCC}12} & \multicolumn{1}{c|}{\cellcolor[HTML]{F4CCCC}13} & \multicolumn{1}{c|}{\cellcolor[HTML]{D9D2E9}}                                   & \multicolumn{1}{c|}{\cellcolor[HTML]{D9D2E9}}                                         \\
        \multirow{-2}{*}{\backslashbox{$y$}{$x$}} & \multicolumn{1}{c|}{\cellcolor[HTML]{FFEBEA}3} & \multicolumn{1}{c|}{\cellcolor[HTML]{FFEBEA}4} & \multicolumn{1}{c|}{\cellcolor[HTML]{FFEBEA}5} & \multicolumn{1}{c|}{\cellcolor[HTML]{FFEBEA}6} & \multicolumn{1}{c|}{\cellcolor[HTML]{FFEBEA}7} & \multicolumn{1}{c|}{\cellcolor[HTML]{FFEBEA}8} & \multicolumn{1}{c|}{\cellcolor[HTML]{FFEBEA}9} & \multicolumn{1}{c|}{\cellcolor[HTML]{FFEBEA}10} & \multicolumn{1}{c|}{\cellcolor[HTML]{FFEBEA}11} & \multicolumn{1}{c|}{\cellcolor[HTML]{FFEBEA}12} & \multicolumn{1}{c|}{\cellcolor[HTML]{FFEBEA}13} & \multicolumn{1}{c|}{\cellcolor[HTML]{FFEBEA}14} & \multicolumn{1}{c|}{\multirow{-2}{*}{\cellcolor[HTML]{D9D2E9}$\widehat{P}(Y)$}} & \multicolumn{1}{c|}{\multirow{-2}{*}{\cellcolor[HTML]{D9D2E9}$y\cdot\widehat{P}(Y)$}} \\ \hline
        \cellcolor[HTML]{CFE2F3}18                                  & \cellcolor[HTML]{C7E9D8}0.0217                 & \cellcolor[HTML]{57BB8A}0.0652                 & \cellcolor[HTML]{FFFFFF}0                      & \cellcolor[HTML]{C7E9D8}0.0217                 & \cellcolor[HTML]{FFFFFF}0                      & \cellcolor[HTML]{C7E9D8}0.0217                 & \cellcolor[HTML]{8FD2B1}0.0435                 & \cellcolor[HTML]{8FD2B1}0.0435                  & \cellcolor[HTML]{C7E9D8}0.0217                  & \cellcolor[HTML]{FFFFFF}0                       & \cellcolor[HTML]{FFFFFF}0                       & \cellcolor[HTML]{FFFFFF}0                       & \cellcolor[HTML]{D9D2E9}0.2391                                                  & \cellcolor[HTML]{D9D2E9}4.3043                                                        \\ \hline
        \cellcolor[HTML]{CFE2F3}19                                  & \cellcolor[HTML]{FFFFFF}0                      & \cellcolor[HTML]{FFFFFF}0                      & \cellcolor[HTML]{C7E9D8}0.0217                 & \cellcolor[HTML]{8FD2B1}0.0435                 & \cellcolor[HTML]{8FD2B1}0.0435                 & \cellcolor[HTML]{C7E9D8}0.0217                 & \cellcolor[HTML]{C7E9D8}0.0217                 & \cellcolor[HTML]{8FD2B1}0.0435                  & \cellcolor[HTML]{FFFFFF}0                       & \cellcolor[HTML]{FFFFFF}0                       & \cellcolor[HTML]{FFFFFF}0                       & \cellcolor[HTML]{C7E9D8}0.0217                  & \cellcolor[HTML]{D9D2E9}0.2174                                                  & \cellcolor[HTML]{D9D2E9}4.1304                                                        \\ \hline
        \rowcolor[HTML]{FFFFFF} 
        \cellcolor[HTML]{CFE2F3}20                                  & 0                                              & 0                                              & 0                                              & 0                                              & 0                                              & 0                                              & 0                                              & 0                                               & 0                                               & 0                                               & 0                                               & 0                                               & \cellcolor[HTML]{D9D2E9}0                                                       & \cellcolor[HTML]{D9D2E9}0                                                             \\ \hline
        \rowcolor[HTML]{FFFFFF} 
        \cellcolor[HTML]{CFE2F3}21                                  & 0                                              & 0                                              & 0                                              & 0                                              & 0                                              & \cellcolor[HTML]{C7E9D8}0.0217                 & \cellcolor[HTML]{C7E9D8}0.0217                 & 0                                               & 0                                               & 0                                               & 0                                               & 0                                               & \cellcolor[HTML]{D9D2E9}0.0435                                                  & \cellcolor[HTML]{D9D2E9}0.913                                                         \\ \hline
        \rowcolor[HTML]{FFFFFF} 
        \cellcolor[HTML]{CFE2F3}22                                  & 0                                              & 0                                              & 0                                              & 0                                              & 0                                              & 0                                              & 0                                              & 0                                               & 0                                               & 0                                               & 0                                               & 0                                               & \cellcolor[HTML]{D9D2E9}0                                                       & \cellcolor[HTML]{D9D2E9}0                                                             \\ \hline
        \rowcolor[HTML]{FFFFFF} 
        \cellcolor[HTML]{CFE2F3}23                                  & \cellcolor[HTML]{C7E9D8}0.0217                 & 0                                              & 0                                              & 0                                              & 0                                              & 0                                              & 0                                              & 0                                               & 0                                               & 0                                               & 0                                               & 0                                               & \cellcolor[HTML]{D9D2E9}0.0217                                                  & \cellcolor[HTML]{D9D2E9}0.5                                                           \\ \hline
        \rowcolor[HTML]{FFFFFF} 
        \cellcolor[HTML]{CFE2F3}24                                  & 0                                              & 0                                              & 0                                              & 0                                              & 0                                              & 0                                              & 0                                              & 0                                               & 0                                               & 0                                               & 0                                               & 0                                               & \cellcolor[HTML]{D9D2E9}0                                                       & \cellcolor[HTML]{D9D2E9}0                                                             \\ \hline
        \rowcolor[HTML]{FFFFFF} 
        \cellcolor[HTML]{CFE2F3}25                                  & 0                                              & 0                                              & 0                                              & 0                                              & 0                                              & 0                                              & 0                                              & 0                                               & 0                                               & 0                                               & 0                                               & 0                                               & \cellcolor[HTML]{D9D2E9}0                                                       & \cellcolor[HTML]{D9D2E9}0                                                             \\ \hline
        \rowcolor[HTML]{FFFFFF} 
        \cellcolor[HTML]{CFE2F3}26                                  & 0                                              & 0                                              & 0                                              & 0                                              & 0                                              & 0                                              & 0                                              & 0                                               & 0                                               & 0                                               & 0                                               & 0                                               & \cellcolor[HTML]{D9D2E9}0                                                       & \cellcolor[HTML]{D9D2E9}0                                                             \\ \hline
        \rowcolor[HTML]{FFFFFF} 
        \cellcolor[HTML]{CFE2F3}27                                  & 0                                              & 0                                              & 0                                              & 0                                              & 0                                              & 0                                              & 0                                              & 0                                               & 0                                               & 0                                               & 0                                               & 0                                               & \cellcolor[HTML]{D9D2E9}0                                                       & \cellcolor[HTML]{D9D2E9}0                                                             \\ \hline
        \rowcolor[HTML]{FFFFFF} 
        \cellcolor[HTML]{CFE2F3}28                                  & 0                                              & 0                                              & 0                                              & 0                                              & 0                                              & 0                                              & 0                                              & 0                                               & 0                                               & 0                                               & 0                                               & 0                                               & \cellcolor[HTML]{D9D2E9}0                                                       & \cellcolor[HTML]{D9D2E9}0                                                             \\ \hline
        \rowcolor[HTML]{FFFFFF} 
        \cellcolor[HTML]{CFE2F3}29                                  & 0                                              & \cellcolor[HTML]{C7E9D8}0.0217                 & 0                                              & 0                                              & 0                                              & 0                                              & 0                                              & 0                                               & 0                                               & 0                                               & 0                                               & 0                                               & \cellcolor[HTML]{D9D2E9}0.0217                                                  & \cellcolor[HTML]{D9D2E9}0.6304                                                        \\ \hline
        \rowcolor[HTML]{FFFFFF} 
        \cellcolor[HTML]{CFE2F3}30                                  & 0                                              & 0                                              & 0                                              & 0                                              & \cellcolor[HTML]{C7E9D8}0.0217                 & 0                                              & 0                                              & 0                                               & 0                                               & 0                                               & 0                                               & 0                                               & \cellcolor[HTML]{D9D2E9}0.0217                                                  & \cellcolor[HTML]{D9D2E9}0.6522                                                        \\ \hline
        \rowcolor[HTML]{FFFFFF} 
        \cellcolor[HTML]{CFE2F3}31                                  & 0                                              & 0                                              & 0                                              & 0                                              & 0                                              & 0                                              & 0                                              & 0                                               & 0                                               & 0                                               & 0                                               & 0                                               & \cellcolor[HTML]{D9D2E9}0                                                       & \cellcolor[HTML]{D9D2E9}0                                                             \\ \hline
        \rowcolor[HTML]{FFFFFF} 
        \cellcolor[HTML]{CFE2F3}32                                  & 0                                              & 0                                              & \cellcolor[HTML]{C7E9D8}0.0217                 & 0                                              & 0                                              & 0                                              & 0                                              & 0                                               & 0                                               & 0                                               & 0                                               & 0                                               & \cellcolor[HTML]{D9D2E9}0.0217                                                  & \cellcolor[HTML]{D9D2E9}0.6957                                                        \\ \hline
        \rowcolor[HTML]{FFFFFF} 
        \cellcolor[HTML]{CFE2F3}33                                  & 0                                              & 0                                              & \cellcolor[HTML]{C7E9D8}0.0217                 & 0                                              & 0                                              & 0                                              & 0                                              & 0                                               & 0                                               & 0                                               & 0                                               & 0                                               & \cellcolor[HTML]{D9D2E9}0.0217                                                  & \cellcolor[HTML]{D9D2E9}0.7173913                                                     \\ \hline
        \rowcolor[HTML]{FFFFFF} 
        \cellcolor[HTML]{CFE2F3}34                                  & 0                                              & 0                                              & 0                                              & 0                                              & 0                                              & \cellcolor[HTML]{8FD2B1}0.0435                 & 0                                              & 0                                               & 0                                               & 0                                               & 0                                               & 0                                               & \cellcolor[HTML]{D9D2E9}0.0435                                                  & \cellcolor[HTML]{D9D2E9}1.4783                                                        \\ \hline
        \rowcolor[HTML]{FFFFFF} 
        \cellcolor[HTML]{CFE2F3}35                                  & 0                                              & 0                                              & 0                                              & 0                                              & 0                                              & 0                                              & 0                                              & 0                                               & 0                                               & 0                                               & 0                                               & 0                                               & \cellcolor[HTML]{D9D2E9}0                                                       & \cellcolor[HTML]{D9D2E9}0                                                             \\ \hline
        \rowcolor[HTML]{FFFFFF} 
        \cellcolor[HTML]{CFE2F3}36                                  & 0                                              & 0                                              & 0                                              & 0                                              & 0                                              & 0                                              & 0                                              & 0                                               & 0                                               & 0                                               & 0                                               & 0                                               & \cellcolor[HTML]{D9D2E9}0                                                       & \cellcolor[HTML]{D9D2E9}0                                                             \\ \hline
        \rowcolor[HTML]{FFFFFF} 
        \cellcolor[HTML]{CFE2F3}37                                  & 0                                              & 0                                              & 0                                              & 0                                              & 0                                              & 0                                              & 0                                              & 0                                               & 0                                               & 0                                               & 0                                               & 0                                               & \cellcolor[HTML]{D9D2E9}0                                                       & \cellcolor[HTML]{D9D2E9}0                                                             \\ \hline
        \rowcolor[HTML]{FFFFFF} 
        \cellcolor[HTML]{CFE2F3}38                                  & 0                                              & 0                                              & \cellcolor[HTML]{C7E9D8}0.0217                 & \cellcolor[HTML]{C7E9D8}0.0217                 & 0                                              & 0                                              & 0                                              & 0                                               & 0                                               & 0                                               & 0                                               & 0                                               & \cellcolor[HTML]{D9D2E9}0.0435                                                  & \cellcolor[HTML]{D9D2E9}1.6522                                                        \\ \hline
        \rowcolor[HTML]{FFFFFF} 
        \cellcolor[HTML]{CFE2F3}39                                  & 0                                              & 0                                              & 0                                              & 0                                              & 0                                              & \cellcolor[HTML]{C7E9D8}0.0217                 & 0                                              & 0                                               & 0                                               & 0                                               & 0                                               & 0                                               & \cellcolor[HTML]{D9D2E9}0.0217                                                  & \cellcolor[HTML]{D9D2E9}0.8478                                                        \\ \hline
        \rowcolor[HTML]{FFFFFF} 
        \cellcolor[HTML]{CFE2F3}40                                  & 0                                              & 0                                              & 0                                              & 0                                              & 0                                              & 0                                              & 0                                              & 0                                               & 0                                               & 0                                               & 0                                               & 0                                               & \cellcolor[HTML]{D9D2E9}0                                                       & \cellcolor[HTML]{D9D2E9}0                                                             \\ \hline
        \rowcolor[HTML]{FFFFFF} 
        \cellcolor[HTML]{CFE2F3}41                                  & 0                                              & 0                                              & 0                                              & \cellcolor[HTML]{C7E9D8}0.0217                 & 0                                              & 0                                              & 0                                              & 0                                               & 0                                               & 0                                               & 0                                               & 0                                               & \cellcolor[HTML]{D9D2E9}0.0217                                                  & \cellcolor[HTML]{D9D2E9}0.8913                                                        \\ \hline
        \rowcolor[HTML]{FFFFFF} 
        \cellcolor[HTML]{CFE2F3}42                                  & 0                                              & 0                                              & 0                                              & 0                                              & 0                                              & 0                                              & 0                                              & 0                                               & 0                                               & 0                                               & 0                                               & 0                                               & \cellcolor[HTML]{D9D2E9}0                                                       & \cellcolor[HTML]{D9D2E9}0                                                             \\ \hline
        \rowcolor[HTML]{FFFFFF} 
        \cellcolor[HTML]{CFE2F3}43                                  & 0                                              & \cellcolor[HTML]{C7E9D8}0.0217                 & 0                                              & 0                                              & 0                                              & 0                                              & 0                                              & 0                                               & 0                                               & 0                                               & 0                                               & 0                                               & \cellcolor[HTML]{D9D2E9}0.0217                                                  & \cellcolor[HTML]{D9D2E9}0.9348                                                        \\ \hline
        \rowcolor[HTML]{FFFFFF} 
        \cellcolor[HTML]{CFE2F3}44                                  & 0                                              & 0                                              & 0                                              & \cellcolor[HTML]{C7E9D8}0.0217                 & 0                                              & 0                                              & 0                                              & 0                                               & 0                                               & 0                                               & 0                                               & 0                                               & \cellcolor[HTML]{D9D2E9}0.0217                                                  & \cellcolor[HTML]{D9D2E9}0.9565217                                                     \\ \hline
        \rowcolor[HTML]{FFFFFF} 
        \cellcolor[HTML]{CFE2F3}45                                  & 0                                              & 0                                              & 0                                              & \cellcolor[HTML]{C7E9D8}0.0217                 & 0                                              & 0                                              & \cellcolor[HTML]{8FD2B1}0.0435                 & 0                                               & 0                                               & 0                                               & 0                                               & 0                                               & \cellcolor[HTML]{D9D2E9}0.0652                                                  & \cellcolor[HTML]{D9D2E9}2.9348                                                        \\ \hline
        \rowcolor[HTML]{FFFFFF} 
        \cellcolor[HTML]{CFE2F3}46                                  & 0                                              & \cellcolor[HTML]{C7E9D8}0.0217                 & 0                                              & 0                                              & 0                                              & 0                                              & 0                                              & 0                                               & 0                                               & 0                                               & 0                                               & 0                                               & \cellcolor[HTML]{D9D2E9}0.0217                                                  & \cellcolor[HTML]{D9D2E9}1.                                                            \\ \hline
        \rowcolor[HTML]{FFFFFF} 
        \cellcolor[HTML]{CFE2F3}47                                  & 0                                              & 0                                              & 0                                              & 0                                              & 0                                              & 0                                              & 0                                              & 0                                               & 0                                               & 0                                               & 0                                               & 0                                               & \cellcolor[HTML]{D9D2E9}0                                                       & \cellcolor[HTML]{D9D2E9}0                                                             \\ \hline
        \rowcolor[HTML]{FFFFFF} 
        \cellcolor[HTML]{CFE2F3}48                                  & \cellcolor[HTML]{C7E9D8}0.0217                 & 0                                              & 0                                              & 0                                              & 0                                              & 0                                              & 0                                              & 0                                               & 0                                               & 0                                               & 0                                               & 0                                               & \cellcolor[HTML]{D9D2E9}0.0217                                                  & \cellcolor[HTML]{D9D2E9}1.0435                                                        \\ \hline
        \rowcolor[HTML]{FFFFFF} 
        \cellcolor[HTML]{CFE2F3}49                                  & 0                                              & 0                                              & \cellcolor[HTML]{C7E9D8}0.0217                 & 0                                              & 0                                              & 0                                              & 0                                              & 0                                               & 0                                               & 0                                               & 0                                               & 0                                               & \cellcolor[HTML]{D9D2E9}0.0217                                                  & \cellcolor[HTML]{D9D2E9}1.0652                                                        \\ \hline
        \rowcolor[HTML]{FFFFFF} 
        \cellcolor[HTML]{CFE2F3}50                                  & 0                                              & 0                                              & 0                                              & \cellcolor[HTML]{C7E9D8}0.0217                 & 0                                              & 0                                              & 0                                              & 0                                               & 0                                               & 0                                               & 0                                               & 0                                               & \cellcolor[HTML]{D9D2E9}0.0217                                                  & \cellcolor[HTML]{D9D2E9}1.087                                                         \\ \hline
        \rowcolor[HTML]{FFFFFF} 
        \cellcolor[HTML]{CFE2F3}51                                  & 0                                              & 0                                              & 0                                              & 0                                              & 0                                              & 0                                              & 0                                              & 0                                               & 0                                               & 0                                               & 0                                               & 0                                               & \cellcolor[HTML]{D9D2E9}0                                                       & \cellcolor[HTML]{D9D2E9}0                                                             \\ \hline
        \rowcolor[HTML]{FFFFFF} 
        \cellcolor[HTML]{CFE2F3}52                                  & 0                                              & 0                                              & 0                                              & 0                                              & 0                                              & 0                                              & 0                                              & 0                                               & 0                                               & 0                                               & 0                                               & 0                                               & \cellcolor[HTML]{D9D2E9}0                                                       & \cellcolor[HTML]{D9D2E9}0                                                             \\ \hline
        \rowcolor[HTML]{FFFFFF} 
        \cellcolor[HTML]{CFE2F3}53                                  & 0                                              & 0                                              & 0                                              & 0                                              & 0                                              & 0                                              & 0                                              & 0                                               & 0                                               & 0                                               & 0                                               & 0                                               & \cellcolor[HTML]{D9D2E9}0                                                       & \cellcolor[HTML]{D9D2E9}0                                                             \\ \hline
        \rowcolor[HTML]{FFFFFF} 
        \cellcolor[HTML]{CFE2F3}54                                  & 0                                              & 0                                              & 0                                              & 0                                              & 0                                              & 0                                              & 0                                              & 0                                               & 0                                               & 0                                               & 0                                               & 0                                               & \cellcolor[HTML]{D9D2E9}0                                                       & \cellcolor[HTML]{D9D2E9}0                                                             \\ \hline
        \rowcolor[HTML]{FFFFFF} 
        \cellcolor[HTML]{CFE2F3}55                                  & 0                                              & 0                                              & 0                                              & 0                                              & \cellcolor[HTML]{C7E9D8}0.0217                 & 0                                              & 0                                              & 0                                               & 0                                               & 0                                               & 0                                               & 0                                               & \cellcolor[HTML]{D9D2E9}0.0217                                                  & \cellcolor[HTML]{D9D2E9}1.19565217                                                    \\ \hline
        \rowcolor[HTML]{FFFFFF} 
        \cellcolor[HTML]{CFE2F3}56                                  & 0                                              & 0                                              & 0                                              & 0                                              & 0                                              & 0                                              & 0                                              & 0                                               & 0                                               & 0                                               & 0                                               & 0                                               & \cellcolor[HTML]{D9D2E9}0                                                       & \cellcolor[HTML]{D9D2E9}0                                                             \\ \hline
        \rowcolor[HTML]{FFFFFF} 
        \cellcolor[HTML]{CFE2F3}57                                  & 0                                              & \cellcolor[HTML]{C7E9D8}0.0217                 & \cellcolor[HTML]{C7E9D8}0.0217                 & 0                                              & 0                                              & 0                                              & 0                                              & 0                                               & 0                                               & 0                                               & 0                                               & 0                                               & \cellcolor[HTML]{D9D2E9}0.0435                                                  & \cellcolor[HTML]{D9D2E9}2.4783                                                        \\ \hline
        \rowcolor[HTML]{D9EAD3} 
        $\widehat{P}(X)$                                            & 0.0652                                         & 0.1522                                         & 0.1304                                         & 0.1739                                         & 0.087                                          & 0.1304                                         & 0.1304                                         & 0.087                                           & 0.0217                                          & 0                                               & 0                                               & 0.0217                                          & \multicolumn{1}{l|}{\cellcolor[HTML]{FCE5CD}$\widehat{E}(X)$}                   & \cellcolor[HTML]{FCE5CD}6.23913                                                       \\ \hline
        \rowcolor[HTML]{D9EAD3} 
        midpoint                                                    & 2.5000                                         & 3.5000                                         & 4.5000                                         & 5.5000                                         & 6.5000                                         & 7.5000                                         & 8.5000                                         & 9.5000                                          & 10.5000                                         & 11.5000                                         & 12.5000                                         & 13.5000                                         & \multicolumn{1}{l|}{\cellcolor[HTML]{FCE5CD}$\widehat{E}(Y)$}                   & \cellcolor[HTML]{FCE5CD}30.10870                                                      \\ \hline
        \cellcolor[HTML]{D9EAD3}$\mathrm{mid}\cdot\widehat{P}(X)$   & \cellcolor[HTML]{D9EAD3}0.163                  & \cellcolor[HTML]{D9EAD3}0.5326                 & \cellcolor[HTML]{D9EAD3}0.587                  & \cellcolor[HTML]{D9EAD3}0.9565217              & \cellcolor[HTML]{D9EAD3}0.5652                 & \cellcolor[HTML]{D9EAD3}0.9783                 & \cellcolor[HTML]{D9EAD3}1.1087                 & \cellcolor[HTML]{D9EAD3}0.8261                  & \cellcolor[HTML]{D9EAD3}0.2283                  & \cellcolor[HTML]{D9EAD3}0                       & \cellcolor[HTML]{D9EAD3}0                       & \cellcolor[HTML]{D9EAD3}0.2935                  & \multicolumn{1}{l|}{}                                                           & \multicolumn{1}{l|}{}                                                                 \\ \hline
        \end{tabular}
    \end{adjustbox}
\end{table}
\noindent The individual probability from the table comes from dividing frequency with the number of samples. $P[X]$ comes from summing the data for all columns. $P[Y]$ comes from summing the data for all rows. The expected value of $X$ $(E[X])$ is 6.23913 which is calculated by summing all of $P[X]$ and the expected value of $Y$ $(E[Y])$ is 30.10870 which is calculated by summing all of $P[Y]$.
\begin{figure}[ht!]
    \begin{tikzpicture}[scale=0.51]
        \begin{axis}
            [
                xlabel = screen duration (Hrs),
                ylabel = \(\widehat{P}(X)\),
                width = 0.9\textwidth,
                grid,
                ybar,
                ymin = 0,
                xmin = 2, xmax = 14,
                xtick distance = 1,
                % ytick distance = 0.05,
                title = \(\widehat{P}(X)\),
                % nodes near coords style={font=\tiny}
            ]
            \addplot+[ybar interval] table [y=Px_SA, col sep=comma, x expr=\coordindex+2] {6 graph.csv};
        \end{axis}
    \end{tikzpicture}
    \begin{tikzpicture}[scale=0.43.5]
        \begin{axis}
            [
                xlabel = Age,
                ylabel = \(\widehat{P}(Y)\),
                width = 1.05\textwidth,
                grid,
                ybar,
                ymin = 0,
                xmin = 17, xmax = 58,
                % xtick = data,
                % xticklabels = {18, 20, 22, 24, 26, 28, 30, 32, 34, 36, 38, 40, 42, 44, 46, 48, 50, 52, 54, 56},
                xtick distance =2,
                % ytick distance = 0.025,
                title = \(\widehat{P}(Y)\),
                % nodes near coords style={font=\tiny}
            ]
            \addplot+[ybar] table [y=Py_SA, col sep=comma, x expr=\coordindex+18] {6 graph.csv};
        \end{axis}
    \end{tikzpicture}
\end{figure}\\
The correlation coefficient is computed by using the formula 
\begin{displaymath}
    \widehat{\rho}_{X, Y} = \dfrac{\widehat{S}_{X, Y}}{s_Xs_Y} \enskip \& \enskip \widehat{S}_{X, Y}=\dfrac{1}{n-1}\dsl^n_{i=1}\left(x_i-\overline{X}\right)\left(y_i-\overline{Y}\right)
\end{displaymath}
In this case, \(\overline{X} = 6.271\), \(\overline{Y} = 30.1087, n = 46\), \(s_x = 2.4862\), \(s_y = 13.2031\). Therefore, \(\widehat{S}_{X, Y} = -10.9625\) and \(\widehat{\rho}_{X, Y} = -0.3340\). Thus, the relation between $X$ and $Y$ has a weak negative relationship.
\subsection*{Screen time vs Sleeping time}
\begin{table}[hb]
    \begin{adjustbox}{width = \textwidth, center}
        \begin{tabular}{|cc|r|r|r|r|r|r|r|r|r|r|r|r|r|r|r|}
            \hline
            \multicolumn{2}{|c|}{}                                                          & \multicolumn{1}{c|}{\cellcolor[HTML]{F4CCCC}2} & \multicolumn{1}{c|}{\cellcolor[HTML]{F4CCCC}3} & \multicolumn{1}{c|}{\cellcolor[HTML]{F4CCCC}4} & \multicolumn{1}{c|}{\cellcolor[HTML]{F4CCCC}5} & \multicolumn{1}{c|}{\cellcolor[HTML]{F4CCCC}6} & \multicolumn{1}{c|}{\cellcolor[HTML]{F4CCCC}7} & \multicolumn{1}{c|}{\cellcolor[HTML]{F4CCCC}8} & \multicolumn{1}{c|}{\cellcolor[HTML]{F4CCCC}9}  & \multicolumn{1}{c|}{\cellcolor[HTML]{F4CCCC}10} & \multicolumn{1}{c|}{\cellcolor[HTML]{F4CCCC}11} & \multicolumn{1}{c|}{\cellcolor[HTML]{F4CCCC}12} & \multicolumn{1}{c|}{\cellcolor[HTML]{F4CCCC}13} & \multicolumn{1}{c|}{\cellcolor[HTML]{D9D2E9}}                                   & \multicolumn{1}{c|}{\cellcolor[HTML]{D9D2E9}}                           & \multicolumn{1}{c|}{\cellcolor[HTML]{D9D2E9}}                                         \\
            \multicolumn{2}{|c|}{\multirow{-2}{*}{\backslashbox{$y$}{$x$}}}                 & \multicolumn{1}{c|}{\cellcolor[HTML]{FFEBEA}3} & \multicolumn{1}{c|}{\cellcolor[HTML]{FFEBEA}4} & \multicolumn{1}{c|}{\cellcolor[HTML]{FFEBEA}5} & \multicolumn{1}{c|}{\cellcolor[HTML]{FFEBEA}6} & \multicolumn{1}{c|}{\cellcolor[HTML]{FFEBEA}7} & \multicolumn{1}{c|}{\cellcolor[HTML]{FFEBEA}8} & \multicolumn{1}{c|}{\cellcolor[HTML]{FFEBEA}9} & \multicolumn{1}{c|}{\cellcolor[HTML]{FFEBEA}10} & \multicolumn{1}{c|}{\cellcolor[HTML]{FFEBEA}11} & \multicolumn{1}{c|}{\cellcolor[HTML]{FFEBEA}12} & \multicolumn{1}{c|}{\cellcolor[HTML]{FFEBEA}13} & \multicolumn{1}{c|}{\cellcolor[HTML]{FFEBEA}14} & \multicolumn{1}{c|}{\multirow{-2}{*}{\cellcolor[HTML]{D9D2E9}$\widehat{P}(Y)$}} & \multicolumn{1}{c|}{\multirow{-2}{*}{\cellcolor[HTML]{D9D2E9}midpoint}} & \multicolumn{1}{c|}{\multirow{-2}{*}{\cellcolor[HTML]{D9D2E9}$\mathrm{mid}\cdot\widehat{P}(Y)$}} \\ \hline
            \rowcolor[HTML]{FFFFFF} 
            \cellcolor[HTML]{C9DAF8}\hspace{7pt}4\hspace{7pt}                 & \cellcolor[HTML]{EBF1FC}5              & 0                                              & 0                                              & 0                                              & 0                                              & 0                                              & 0                                              & 0                                              & \cellcolor[HTML]{BCE4D1}0.0435                  & 0                                               & 0                                               & 0                                               & 0                                               & \cellcolor[HTML]{D9D2E9}0.0435                                                  & \cellcolor[HTML]{D9D2E9}4.5                                             & \cellcolor[HTML]{D9D2E9}0.1957                                                        \\ \hline
            \rowcolor[HTML]{FFFFFF} 
            \cellcolor[HTML]{C9DAF8}5              & \cellcolor[HTML]{EBF1FC}6              & 0                                              & 0                                              & 0                                              & \cellcolor[HTML]{DEF2E8}0.0217                 & 0                                              & 0                                              & 0                                              & 0                                               & 0                                               & 0                                               & 0                                               & 0                                               & \cellcolor[HTML]{D9D2E9}0.0217                                                  & \cellcolor[HTML]{D9D2E9}5.5                                             & \cellcolor[HTML]{D9D2E9}0.1195652                                                     \\ \hline
            \cellcolor[HTML]{C9DAF8}6              & \cellcolor[HTML]{EBF1FC}7              & \cellcolor[HTML]{FFFFFF}0                      & \cellcolor[HTML]{DEF2E8}0.0217                 & \cellcolor[HTML]{DEF2E8}0.0217                 & \cellcolor[HTML]{BCE4D1}0.0435                 & \cellcolor[HTML]{9BD7B9}0.0652                 & \cellcolor[HTML]{BCE4D1}0.0435                 & \cellcolor[HTML]{9BD7B9}0.0652                 & \cellcolor[HTML]{FFFFFF}0                       & \cellcolor[HTML]{DEF2E8}0.0217                  & \cellcolor[HTML]{FFFFFF}0                       & \cellcolor[HTML]{FFFFFF}0                       & \cellcolor[HTML]{FFFFFF}0                       & \cellcolor[HTML]{D9D2E9}0.2826                                                  & \cellcolor[HTML]{D9D2E9}6.5                                             & \cellcolor[HTML]{D9D2E9}1.837                                                         \\ \hline
            \cellcolor[HTML]{C9DAF8}7              & \cellcolor[HTML]{EBF1FC}8              & \cellcolor[HTML]{DEF2E8}0.0217                 & \cellcolor[HTML]{9BD7B9}0.0652                 & \cellcolor[HTML]{57BB8A}0.1087                 & \cellcolor[HTML]{79C9A2}0.087                  & \cellcolor[HTML]{DEF2E8}0.0217                 & \cellcolor[HTML]{9BD7B9}0.0652                 & \cellcolor[HTML]{BCE4D1}0.0435                 & \cellcolor[HTML]{FFFFFF}0                       & \cellcolor[HTML]{FFFFFF}0                       & \cellcolor[HTML]{FFFFFF}0                       & \cellcolor[HTML]{FFFFFF}0                       & \cellcolor[HTML]{FFFFFF}0                       & \cellcolor[HTML]{D9D2E9}0.413                                                   & \cellcolor[HTML]{D9D2E9}7.5                                             & \cellcolor[HTML]{D9D2E9}3.0978                                                        \\ \hline
            \cellcolor[HTML]{C9DAF8}8              & \cellcolor[HTML]{EBF1FC}9              & \cellcolor[HTML]{BCE4D1}0.0435                 & \cellcolor[HTML]{BCE4D1}0.0435                 & \cellcolor[HTML]{FFFFFF}0                      & \cellcolor[HTML]{DEF2E8}0.0217                 & \cellcolor[HTML]{FFFFFF}0                      & \cellcolor[HTML]{DEF2E8}0.0217                 & \cellcolor[HTML]{DEF2E8}0.0217                 & \cellcolor[HTML]{BCE4D1}0.0435                  & \cellcolor[HTML]{FFFFFF}0                       & \cellcolor[HTML]{FFFFFF}0                       & \cellcolor[HTML]{FFFFFF}0                       & \cellcolor[HTML]{FFFFFF}0                       & \cellcolor[HTML]{D9D2E9}0.1956522                                               & \cellcolor[HTML]{D9D2E9}8.5                                             & \cellcolor[HTML]{D9D2E9}1.6630435                                                     \\ \hline
            \rowcolor[HTML]{FFFFFF} 
            \cellcolor[HTML]{C9DAF8}9              & \cellcolor[HTML]{EBF1FC}10             & 0                                              & \cellcolor[HTML]{DEF2E8}0.0217                 & 0                                              & 0                                              & 0                                              & 0                                              & 0                                              & 0                                               & 0                                               & 0                                               & 0                                               & \cellcolor[HTML]{DEF2E8}0.0217                  & \cellcolor[HTML]{D9D2E9}0.0435                                                  & \cellcolor[HTML]{D9D2E9}9.5                                             & \cellcolor[HTML]{D9D2E9}0.413                                                         \\ \hline
            \multicolumn{2}{|c|}{\cellcolor[HTML]{FFF2CC}$\widehat{P}(X)$}                  & \cellcolor[HTML]{FFF2CC}0.0652                 & \cellcolor[HTML]{FFF2CC}0.1522                 & \cellcolor[HTML]{FFF2CC}0.1304                 & \cellcolor[HTML]{FFF2CC}0.1739                 & \cellcolor[HTML]{FFF2CC}0.087                  & \cellcolor[HTML]{FFF2CC}0.1304                 & \cellcolor[HTML]{FFF2CC}0.1304                 & \cellcolor[HTML]{FFF2CC}0.087                   & \cellcolor[HTML]{FFF2CC}0.0217                  & \cellcolor[HTML]{FFF2CC}0                       & \cellcolor[HTML]{FFF2CC}0                       & \cellcolor[HTML]{FFF2CC}0.0217                  & \multicolumn{1}{l|}{}                                                           & \multicolumn{1}{l|}{}                                                   & \multicolumn{1}{l|}{}                                                                 \\ \hline
            \multicolumn{2}{|c|}{\cellcolor[HTML]{FFF2CC}midpoint}                          & \cellcolor[HTML]{FFF2CC}2.5                    & \cellcolor[HTML]{FFF2CC}3.5                    & \cellcolor[HTML]{FFF2CC}4.5                    & \cellcolor[HTML]{FFF2CC}5.5                    & \cellcolor[HTML]{FFF2CC}6.5                    & \cellcolor[HTML]{FFF2CC}7.5                    & \cellcolor[HTML]{FFF2CC}8.5                    & \cellcolor[HTML]{FFF2CC}9.5                     & \cellcolor[HTML]{FFF2CC}10.5                    & \cellcolor[HTML]{FFF2CC}11.5                    & \cellcolor[HTML]{FFF2CC}12.5                    & \cellcolor[HTML]{FFF2CC}13.5                    & \multicolumn{1}{l|}{}                                                           & \multicolumn{1}{l|}{\cellcolor[HTML]{E6B8AF}$\widehat{E}(X)$}           & \cellcolor[HTML]{E6B8AF}6.2391                                                        \\ \hline
            \multicolumn{2}{|c|}{\cellcolor[HTML]{FFF2CC}$\mathrm{mid}\cdot\widehat{P}(X)$} & \cellcolor[HTML]{FFF2CC}0.163                  & \cellcolor[HTML]{FFF2CC}0.5326                 & \cellcolor[HTML]{FFF2CC}0.587                  & \cellcolor[HTML]{FFF2CC}0.9565217              & \cellcolor[HTML]{FFF2CC}0.5652                 & \cellcolor[HTML]{FFF2CC}0.9783                 & \cellcolor[HTML]{FFF2CC}1.1087                 & \cellcolor[HTML]{FFF2CC}0.8261                  & \cellcolor[HTML]{FFF2CC}0.2283                  & \cellcolor[HTML]{FFF2CC}0                       & \cellcolor[HTML]{FFF2CC}0                       & \cellcolor[HTML]{FFF2CC}0.2935                  & \multicolumn{1}{l|}{}                                                           & \multicolumn{1}{l|}{\cellcolor[HTML]{E6B8AF}$\widehat{E}(Y)$}           & \cellcolor[HTML]{E6B8AF}7.3261                                                        \\ \hline
        \end{tabular}
    \end{adjustbox}
\end{table}
\noindent The individual probability from the table comes from dividing frequency with the number of samples. $P[X]$ comes from summing the data for all columns. $P[Y]$ comes from summing the data for all rows. The expected value of $X$ $(E[X])$ is 6.23913 which is calculated by summing all of $P[X]$ and the expected value of $Y$ $(E[Y])$ is 7.3261 which is calculated by summing all of $P[Y]$.
\begin{figure}[h!]
    \begin{tikzpicture}[scale=0.5]
        \begin{axis}
            [
                xlabel = screen duration (Hrs),
                ylabel = \(\widehat{P}(X)\),
                width = 14cm,
                height = 10cm,
                grid,
                ybar,
                ymin = 0,
                xmin = 2, xmax = 14,
                xtick distance = 1,
                % ytick distance = 0.05,
                title = \(\widehat{P}(X)\),
                % nodes near coords style={font=\tiny}
            ]
            \addplot+[ybar interval] table [y=Px_SS, col sep=comma, x expr=\coordindex+2] {6 graph.csv};
        \end{axis}
    \end{tikzpicture}
    \begin{tikzpicture}[scale=0.5]
        \begin{axis}
            [
                xlabel = Sleeping time,
                ylabel = \(\widehat{P}(Y)\),
                width = 14cm,
                height = 10cm,
                grid,
                ybar,
                ymin = 0,
                xmin = 4, xmax = 10,
                xtick distance =1,
                % ytick distance = 0.1,
                title = \(\widehat{P}(Y)\),
                % nodes near coords style={font=\tiny}
            ]
            \addplot+[ybar interval] table [y=Py_SS, col sep=comma, x expr=\coordindex+4] {6 graph.csv};
        \end{axis}
    \end{tikzpicture}
\end{figure}\newpage\noindent
The correlation coefficient is computed by using the formula 
\begin{displaymath}
    \widehat{\rho}_{X, Y} = \dfrac{\widehat{S}_{X, Y}}{s_Xs_Y} \enskip \& \enskip \widehat{S}_{X, Y}=\dfrac{1}{n-1}\dsl^n_{i=1}\left(x_i-\overline{X}\right)\left(y_i-\overline{Y}\right)
\end{displaymath}
In this case, \(\overline{X} = 6.271, \overline{Y} = 7.2178, n = 46, s_x = 2.4862, s_y = 1.0068\). Therefore, \(\widehat{S}_{X, Y} = -0.3581\) and \(\widehat{\rho}_{X, Y} = -0.1431\). Thus, the relation between $X$ and $Y$ has a weak negative relationship.\\
\subsection*{Screen time vs Notification}
\begin{table}[ht]
    \begin{adjustbox}{width = \textwidth, center}
        \begin{tabular}{|cc|r|r|r|r|r|r|r|r|r|r|r|r|r|r|r|}
            \hline
            \multicolumn{2}{|c|}{}                                                          & \multicolumn{1}{c|}{\cellcolor[HTML]{F4CCCC}2} & \multicolumn{1}{c|}{\cellcolor[HTML]{F4CCCC}3} & \multicolumn{1}{c|}{\cellcolor[HTML]{F4CCCC}4} & \multicolumn{1}{c|}{\cellcolor[HTML]{F4CCCC}5} & \multicolumn{1}{c|}{\cellcolor[HTML]{F4CCCC}6} & \multicolumn{1}{c|}{\cellcolor[HTML]{F4CCCC}7} & \multicolumn{1}{c|}{\cellcolor[HTML]{F4CCCC}8} & \multicolumn{1}{c|}{\cellcolor[HTML]{F4CCCC}9}  & \multicolumn{1}{c|}{\cellcolor[HTML]{F4CCCC}10} & \multicolumn{1}{c|}{\cellcolor[HTML]{F4CCCC}11} & \multicolumn{1}{c|}{\cellcolor[HTML]{F4CCCC}12} & \multicolumn{1}{c|}{\cellcolor[HTML]{F4CCCC}13} & \multicolumn{1}{c|}{\cellcolor[HTML]{D9D2E9}}                                   & \multicolumn{1}{c|}{\cellcolor[HTML]{D9D2E9}}                           & \multicolumn{1}{c|}{\cellcolor[HTML]{D9D2E9}}                                         \\
            \multicolumn{2}{|c|}{\multirow{-2}{*}{\backslashbox{$y$}{$x$}}}                 & \multicolumn{1}{c|}{\cellcolor[HTML]{FFEBEA}3} & \multicolumn{1}{c|}{\cellcolor[HTML]{FFEBEA}4} & \multicolumn{1}{c|}{\cellcolor[HTML]{FFEBEA}5} & \multicolumn{1}{c|}{\cellcolor[HTML]{FFEBEA}6} & \multicolumn{1}{c|}{\cellcolor[HTML]{FFEBEA}7} & \multicolumn{1}{c|}{\cellcolor[HTML]{FFEBEA}8} & \multicolumn{1}{c|}{\cellcolor[HTML]{FFEBEA}9} & \multicolumn{1}{c|}{\cellcolor[HTML]{FFEBEA}10} & \multicolumn{1}{c|}{\cellcolor[HTML]{FFEBEA}11} & \multicolumn{1}{c|}{\cellcolor[HTML]{FFEBEA}12} & \multicolumn{1}{c|}{\cellcolor[HTML]{FFEBEA}13} & \multicolumn{1}{c|}{\cellcolor[HTML]{FFEBEA}14} & \multicolumn{1}{c|}{\multirow{-2}{*}{\cellcolor[HTML]{D9D2E9}$\widehat{P}(Y)$}} & \multicolumn{1}{c|}{\multirow{-2}{*}{\cellcolor[HTML]{D9D2E9}midpoint}} & \multicolumn{1}{c|}{\multirow{-2}{*}{\cellcolor[HTML]{D9D2E9}$\mathrm{mid}\cdot\widehat{P}(Y)$}} \\ \hline
            \rowcolor[HTML]{FFFFFF} 
            \cellcolor[HTML]{D9EAD3}\hspace{7pt}0\hspace{7pt}              & \cellcolor[HTML]{C8E4BE}25             & \cellcolor[HTML]{C7E9D8}0.0217                 & 0                                              & \cellcolor[HTML]{C7E9D8}0.0217                 & \cellcolor[HTML]{57BB8A}0.0652                 & 0                                              & 0                                              & 0                                              & 0                                               & 0                                               & 0                                               & 0                                               & 0                                               & \cellcolor[HTML]{D9D2E9}0.1087                                                  & \cellcolor[HTML]{D9D2E9}12.5                                            & \cellcolor[HTML]{D9D2E9}1.3587                                                        \\ \hline
            \cellcolor[HTML]{D9EAD3}25             & \cellcolor[HTML]{C8E4BE}50             & \cellcolor[HTML]{FFFFFF}0                      & \cellcolor[HTML]{FFFFFF}0                      & \cellcolor[HTML]{C7E9D8}0.0217                 & \cellcolor[HTML]{FFFFFF}0                      & \cellcolor[HTML]{C7E9D8}0.0217                 & \cellcolor[HTML]{8FD2B1}0.0435                 & \cellcolor[HTML]{C7E9D8}0.0217                 & \cellcolor[HTML]{FFFFFF}0                       & \cellcolor[HTML]{FFFFFF}0                       & \cellcolor[HTML]{FFFFFF}0                       & \cellcolor[HTML]{FFFFFF}0                       & \cellcolor[HTML]{C7E9D8}0.0217                  & \cellcolor[HTML]{D9D2E9}0.1304                                                  & \cellcolor[HTML]{D9D2E9}37.5                                            & \cellcolor[HTML]{D9D2E9}4.8913                                                        \\ \hline
            \cellcolor[HTML]{D9EAD3}50             & \cellcolor[HTML]{C8E4BE}75             & \cellcolor[HTML]{C7E9D8}0.0217                 & \cellcolor[HTML]{C7E9D8}0.0217                 & \cellcolor[HTML]{FFFFFF}0                      & \cellcolor[HTML]{8FD2B1}0.0435                 & \cellcolor[HTML]{FFFFFF}0                      & \cellcolor[HTML]{FFFFFF}0                      & \cellcolor[HTML]{8FD2B1}0.0435                 & \cellcolor[HTML]{FFFFFF}0                       & \cellcolor[HTML]{FFFFFF}0                       & \cellcolor[HTML]{FFFFFF}0                       & \cellcolor[HTML]{FFFFFF}0                       & \cellcolor[HTML]{FFFFFF}0                       & \cellcolor[HTML]{D9D2E9}0.1304                                                  & \cellcolor[HTML]{D9D2E9}62.5                                            & \cellcolor[HTML]{D9D2E9}8.1522                                                        \\ \hline
            \cellcolor[HTML]{D9EAD3}75             & \cellcolor[HTML]{C8E4BE}100            & \cellcolor[HTML]{C7E9D8}0.0217                 & \cellcolor[HTML]{57BB8A}0.0652                 & \cellcolor[HTML]{C7E9D8}0.0217                 & \cellcolor[HTML]{8FD2B1}0.0435                 & \cellcolor[HTML]{8FD2B1}0.0435                 & \cellcolor[HTML]{FFFFFF}0                      & \cellcolor[HTML]{FFFFFF}0                      & \cellcolor[HTML]{FFFFFF}0                       & \cellcolor[HTML]{FFFFFF}0                       & \cellcolor[HTML]{FFFFFF}0                       & \cellcolor[HTML]{FFFFFF}0                       & \cellcolor[HTML]{FFFFFF}0                       & \cellcolor[HTML]{D9D2E9}0.1956522                                               & \cellcolor[HTML]{D9D2E9}87.5                                            & \cellcolor[HTML]{D9D2E9}17.1195652                                                    \\ \hline
            \cellcolor[HTML]{D9EAD3}100            & \cellcolor[HTML]{C8E4BE}125            & \cellcolor[HTML]{FFFFFF}0                      & \cellcolor[HTML]{8FD2B1}0.0435                 & \cellcolor[HTML]{C7E9D8}0.0217                 & \cellcolor[HTML]{C7E9D8}0.0217                 & \cellcolor[HTML]{C7E9D8}0.0217                 & \cellcolor[HTML]{C7E9D8}0.0217                 & \cellcolor[HTML]{FFFFFF}0                      & \cellcolor[HTML]{C7E9D8}0.0217                  & \cellcolor[HTML]{FFFFFF}0                       & \cellcolor[HTML]{FFFFFF}0                       & \cellcolor[HTML]{FFFFFF}0                       & \cellcolor[HTML]{FFFFFF}0                       & \cellcolor[HTML]{D9D2E9}0.1522                                                  & \cellcolor[HTML]{D9D2E9}112.5                                           & \cellcolor[HTML]{D9D2E9}17.1196                                                       \\ \hline
            \rowcolor[HTML]{FFFFFF} 
            \cellcolor[HTML]{D9EAD3}125            & \cellcolor[HTML]{C8E4BE}150            & 0                                              & 0                                              & \cellcolor[HTML]{C7E9D8}0.0217                 & 0                                              & 0                                              & 0                                              & \cellcolor[HTML]{C7E9D8}0.0217                 & \cellcolor[HTML]{8FD2B1}0.0435                  & 0                                               & 0                                               & 0                                               & 0                                               & \cellcolor[HTML]{D9D2E9}0.087                                                   & \cellcolor[HTML]{D9D2E9}137.5                                           & \cellcolor[HTML]{D9D2E9}11.95652174                                                   \\ \hline
            \rowcolor[HTML]{FFFFFF} 
            \cellcolor[HTML]{D9EAD3}150            & \cellcolor[HTML]{C8E4BE}175            & 0                                              & 0                                              & 0                                              & 0                                              & 0                                              & \cellcolor[HTML]{57BB8A}0.0652                 & 0                                              & 0                                               & \cellcolor[HTML]{C7E9D8}0.0217                  & 0                                               & 0                                               & 0                                               & \cellcolor[HTML]{D9D2E9}0.087                                                   & \cellcolor[HTML]{D9D2E9}162.5                                           & \cellcolor[HTML]{D9D2E9}14.1304                                                       \\ \hline
            \rowcolor[HTML]{FFFFFF} 
            \cellcolor[HTML]{D9EAD3}175            & \cellcolor[HTML]{C8E4BE}200            & 0                                              & 0                                              & 0                                              & 0                                              & 0                                              & 0                                              & \cellcolor[HTML]{8FD2B1}0.0435                 & 0                                               & 0                                               & 0                                               & 0                                               & 0                                               & \cellcolor[HTML]{D9D2E9}0.0435                                                  & \cellcolor[HTML]{D9D2E9}187.5                                           & \cellcolor[HTML]{D9D2E9}8.1522                                                        \\ \hline
            \rowcolor[HTML]{FFFFFF} 
            \cellcolor[HTML]{D9EAD3}200            & \cellcolor[HTML]{C8E4BE}225            & 0                                              & \cellcolor[HTML]{C7E9D8}0.0217                 & \cellcolor[HTML]{C7E9D8}0.0217                 & 0                                              & 0                                              & 0                                              & 0                                              & \cellcolor[HTML]{C7E9D8}0.0217                  & 0                                               & 0                                               & 0                                               & 0                                               & \cellcolor[HTML]{D9D2E9}0.0652                                                  & \cellcolor[HTML]{D9D2E9}212.5                                           & \cellcolor[HTML]{D9D2E9}13.8587                                                       \\ \hline
            \multicolumn{2}{|c|}{\cellcolor[HTML]{FCE5CD}$\widehat{P}(X)$}                  & \cellcolor[HTML]{FCE5CD}0.0652                 & \cellcolor[HTML]{FCE5CD}0.1522                 & \cellcolor[HTML]{FCE5CD}0.1304                 & \cellcolor[HTML]{FCE5CD}0.1739                 & \cellcolor[HTML]{FCE5CD}0.087                  & \cellcolor[HTML]{FCE5CD}0.1304                 & \cellcolor[HTML]{FCE5CD}0.1304                 & \cellcolor[HTML]{FCE5CD}0.087                   & \cellcolor[HTML]{FCE5CD}0.0217                  & \cellcolor[HTML]{FCE5CD}0                       & \cellcolor[HTML]{FCE5CD}0                       & \cellcolor[HTML]{FCE5CD}0.0217                  & \multicolumn{1}{l|}{}                                                           & \multicolumn{1}{l|}{}                                                   & \multicolumn{1}{l|}{}                                                                 \\ \hline
            \multicolumn{2}{|c|}{\cellcolor[HTML]{FCE5CD}midpoint}                          & \cellcolor[HTML]{FCE5CD}2.5                    & \cellcolor[HTML]{FCE5CD}3.5                    & \cellcolor[HTML]{FCE5CD}4.5                    & \cellcolor[HTML]{FCE5CD}5.5                    & \cellcolor[HTML]{FCE5CD}6.5                    & \cellcolor[HTML]{FCE5CD}7.5                    & \cellcolor[HTML]{FCE5CD}8.5                    & \cellcolor[HTML]{FCE5CD}9.5                     & \cellcolor[HTML]{FCE5CD}10.5                    & \cellcolor[HTML]{FCE5CD}11.5                    & \cellcolor[HTML]{FCE5CD}12.5                    & \cellcolor[HTML]{FCE5CD}13.5                    & \multicolumn{1}{l|}{}                                                           & \multicolumn{1}{l|}{\cellcolor[HTML]{D0E0E3}$\widehat{E}(Y)$}           & \cellcolor[HTML]{D0E0E3}96.7391                                                       \\ \hline
            \multicolumn{2}{|c|}{\cellcolor[HTML]{FCE5CD}$\mathrm{mid}\cdot\widehat{P}(X)$} & \cellcolor[HTML]{FCE5CD}0.163                  & \cellcolor[HTML]{FCE5CD}0.5326                 & \cellcolor[HTML]{FCE5CD}0.587                  & \cellcolor[HTML]{FCE5CD}0.9565                 & \cellcolor[HTML]{FCE5CD}0.5652                 & \cellcolor[HTML]{FCE5CD}0.9783                 & \cellcolor[HTML]{FCE5CD}1.1087                 & \cellcolor[HTML]{FCE5CD}0.8261                  & \cellcolor[HTML]{FCE5CD}0.2283                  & \cellcolor[HTML]{FCE5CD}0                       & \cellcolor[HTML]{FCE5CD}0                       & \cellcolor[HTML]{FCE5CD}0.2935                  & \multicolumn{1}{l|}{}                                                           & \multicolumn{1}{l|}{\cellcolor[HTML]{D0E0E3}$\widehat{E}(X)$}           & \cellcolor[HTML]{D0E0E3}6.2391                                                        \\ \hline
            \end{tabular}
    \end{adjustbox}
\end{table}\\
\noindent The individual probability from the table comes from dividing frequency with the number of samples. $P[X]$ comes from summing the data for all columns. $P[Y]$ comes from summing the data for all rows. The expected value of $X$ $(E[X])$ is 6.23913 which is calculated by summing all of $P[X]$ and the expected value of $Y$ $(E[Y])$ is 81.5217 which is calculated by summing all of $P[Y]$.
\begin{figure}[ht!]
    \begin{tikzpicture}[scale=0.51]
        \begin{axis}
            [
                xlabel = screen duration (Hrs),
                ylabel = \(\widehat{P}(X)\),
                width = 0.9\textwidth,
                grid,
                ybar,
                ymin = 0,
                xmin = 2, xmax = 14,
                xtick distance = 1,
                % ytick distance = 0.05,
                title = \(\widehat{P}(X)\),
                % nodes near coords style={font=\tiny}
            ]
            \addplot+[ybar interval] table [y=Px_SA, col sep=comma, x expr=\coordindex+2] {6 graph.csv};
        \end{axis}
    \end{tikzpicture}
    \begin{tikzpicture}[scale=0.51]
        \begin{axis}
            [
                xlabel = Notification,
                ylabel = \(\widehat{P}(Y)\),
                width = 0.9\textwidth,
                grid,
                ybar,
                ymin = 0,
                xmin = 0, xmax = 9,
                xtick = data,
                xticklabels = {0, 25, 50, 75, 100, 125, 150, 175, 200, 225},
                % xtick distance = 25,
                % ytick distance = 0.025,
                title = \(\widehat{P}(Y)\),
                % nodes near coords style={font=\tiny}
            ]
            \addplot+[ybar interval] table [y=Py_SN, col sep=comma, x expr=\coordindex] {6 graph.csv};
        \end{axis}
    \end{tikzpicture}
\end{figure}\\
The correlation coefficient is computed by using the formula 
\begin{displaymath}
    \widehat{\rho}_{X, Y} = \dfrac{\widehat{S}_{X, Y}}{s_Xs_Y} \enskip \& \enskip \widehat{S}_{X, Y}=\dfrac{1}{n-1}\dsl^n_{i=1}\left(x_i-\overline{X}\right)\left(y_i-\overline{Y}\right)
\end{displaymath}
In this case, \(\overline{X} = 6.271, \overline{Y} = 80.7174, n = 46, s_x = 2.4862, s_y = 10.8805\). Therefore, \(\widehat{S}_{X, Y} = -1.3246\) and \(\widehat{\rho}_{X, Y} = -0.0490\). Thus, the relation between $X$ and $Y$ has a weak negative relationship.\newpage
\subsection*{Screen time vs Heart rate}
\begin{table}[hb]
    \begin{adjustbox}{width = \textwidth, center}
        \begin{tabular}{|cc|r|r|r|r|r|r|r|r|r|r|r|r|r|r|r|}
            \hline
            \multicolumn{2}{|c|}{}                                                          & \multicolumn{1}{c|}{\cellcolor[HTML]{F4CCCC}2} & \multicolumn{1}{c|}{\cellcolor[HTML]{F4CCCC}3} & \multicolumn{1}{c|}{\cellcolor[HTML]{F4CCCC}4} & \multicolumn{1}{c|}{\cellcolor[HTML]{F4CCCC}5} & \multicolumn{1}{c|}{\cellcolor[HTML]{F4CCCC}6} & \multicolumn{1}{c|}{\cellcolor[HTML]{F4CCCC}7} & \multicolumn{1}{c|}{\cellcolor[HTML]{F4CCCC}8} & \multicolumn{1}{c|}{\cellcolor[HTML]{F4CCCC}9}  & \multicolumn{1}{c|}{\cellcolor[HTML]{F4CCCC}10} & \multicolumn{1}{c|}{\cellcolor[HTML]{F4CCCC}11} & \multicolumn{1}{c|}{\cellcolor[HTML]{F4CCCC}12} & \multicolumn{1}{c|}{\cellcolor[HTML]{F4CCCC}13} & \multicolumn{1}{c|}{\cellcolor[HTML]{D9D2E9}}                                   & \multicolumn{1}{c|}{\cellcolor[HTML]{D9D2E9}}                           & \multicolumn{1}{c|}{\cellcolor[HTML]{D9D2E9}}                                                    \\
            \multicolumn{2}{|c|}{\multirow{-2}{*}{\backslashbox{$y$}{$x$}}}                 & \multicolumn{1}{c|}{\cellcolor[HTML]{FFEBEA}3} & \multicolumn{1}{c|}{\cellcolor[HTML]{FFEBEA}4} & \multicolumn{1}{c|}{\cellcolor[HTML]{FFEBEA}5} & \multicolumn{1}{c|}{\cellcolor[HTML]{FFEBEA}6} & \multicolumn{1}{c|}{\cellcolor[HTML]{FFEBEA}7} & \multicolumn{1}{c|}{\cellcolor[HTML]{FFEBEA}8} & \multicolumn{1}{c|}{\cellcolor[HTML]{FFEBEA}9} & \multicolumn{1}{c|}{\cellcolor[HTML]{FFEBEA}10} & \multicolumn{1}{c|}{\cellcolor[HTML]{FFEBEA}11} & \multicolumn{1}{c|}{\cellcolor[HTML]{FFEBEA}12} & \multicolumn{1}{c|}{\cellcolor[HTML]{FFEBEA}13} & \multicolumn{1}{c|}{\cellcolor[HTML]{FFEBEA}14} & \multicolumn{1}{c|}{\multirow{-2}{*}{\cellcolor[HTML]{D9D2E9}$\widehat{P}(Y)$}} & \multicolumn{1}{c|}{\multirow{-2}{*}{\cellcolor[HTML]{D9D2E9}midpoint}} & \multicolumn{1}{c|}{\multirow{-2}{*}{\cellcolor[HTML]{D9D2E9}$\mathrm{mid}\cdot\widehat{P}(Y)$}} \\ \hline
            \rowcolor[HTML]{FFFFFF} 
            \cellcolor[HTML]{C8E4BE}\hspace{4.5pt}50\hspace{4.5pt}             & \cellcolor[HTML]{D9EAD3}55             & 0                                              & 0                                              & 0                                              & 0                                              & 0                                              & 0                                              & \cellcolor[HTML]{C7E9D8}0.0217                 & 0                                               & 0                                               & 0                                               & 0                                               & 0                                               & \cellcolor[HTML]{D9D2E9}0.0217                                                  & \cellcolor[HTML]{D9D2E9}52.5                                            & \cellcolor[HTML]{D9D2E9}1.1413                                                                   \\ \hline
            \rowcolor[HTML]{FFFFFF} 
            \cellcolor[HTML]{C8E4BE}55             & \cellcolor[HTML]{D9EAD3}60             & 0                                              & \cellcolor[HTML]{C7E9D8}0.0217                 & 0                                              & 0                                              & 0                                              & 0                                              & 0                                              & 0                                               & 0                                               & 0                                               & 0                                               & 0                                               & \cellcolor[HTML]{D9D2E9}0.0217                                                  & \cellcolor[HTML]{D9D2E9}57.5                                            & \cellcolor[HTML]{D9D2E9}1.25                                                                     \\ \hline
            \rowcolor[HTML]{FFFFFF} 
            \cellcolor[HTML]{C8E4BE}60             & \cellcolor[HTML]{D9EAD3}65             & 0                                              & 0                                              & 0                                              & 0                                              & 0                                              & \cellcolor[HTML]{8FD2B1}0.0435                 & 0                                              & 0                                               & 0                                               & 0                                               & 0                                               & 0                                               & \cellcolor[HTML]{D9D2E9}0.0435                                                  & \cellcolor[HTML]{D9D2E9}62.5                                            & \cellcolor[HTML]{D9D2E9}2.7174                                                                   \\ \hline
            \cellcolor[HTML]{C8E4BE}65             & \cellcolor[HTML]{D9EAD3}70             & \cellcolor[HTML]{FFFFFF}0                      & \cellcolor[HTML]{FFFFFF}0                      & \cellcolor[HTML]{FFFFFF}0                      & \cellcolor[HTML]{FFFFFF}0                      & \cellcolor[HTML]{C7E9D8}0.0217                 & \cellcolor[HTML]{C7E9D8}0.0217                 & \cellcolor[HTML]{C7E9D8}0.0217                 & \cellcolor[HTML]{C7E9D8}0.0217                  & \cellcolor[HTML]{FFFFFF}0                       & \cellcolor[HTML]{FFFFFF}0                       & \cellcolor[HTML]{FFFFFF}0                       & \cellcolor[HTML]{FFFFFF}0                       & \cellcolor[HTML]{D9D2E9}0.087                                                   & \cellcolor[HTML]{D9D2E9}67.5                                            & \cellcolor[HTML]{D9D2E9}5.8696                                                                   \\ \hline
            \rowcolor[HTML]{FFFFFF} 
            \cellcolor[HTML]{C8E4BE}70             & \cellcolor[HTML]{D9EAD3}75             & 0                                              & \cellcolor[HTML]{C7E9D8}0.0217                 & 0                                              & \cellcolor[HTML]{C7E9D8}0.0217                 & 0                                              & 0                                              & 0                                              & \cellcolor[HTML]{C7E9D8}0.0217                  & 0                                               & 0                                               & 0                                               & 0                                               & \cellcolor[HTML]{D9D2E9}0.0652                                                  & \cellcolor[HTML]{D9D2E9}72.5                                            & \cellcolor[HTML]{D9D2E9}4.7283                                                                   \\ \hline
            \rowcolor[HTML]{FFFFFF} 
            \cellcolor[HTML]{C8E4BE}75             & \cellcolor[HTML]{D9EAD3}80             & 0                                              & \cellcolor[HTML]{8FD2B1}0.0435                 & \cellcolor[HTML]{C7E9D8}0.0217                 & 0                                              & 0                                              & 0                                              & 0                                              & 0                                               & 0                                               & 0                                               & 0                                               & 0                                               & \cellcolor[HTML]{D9D2E9}0.0652                                                  & \cellcolor[HTML]{D9D2E9}77.5                                            & \cellcolor[HTML]{D9D2E9}5.0543                                                                   \\ \hline
            \cellcolor[HTML]{C8E4BE}80             & \cellcolor[HTML]{D9EAD3}85             & \cellcolor[HTML]{C7E9D8}0.0217                 & \cellcolor[HTML]{8FD2B1}0.0435                 & \cellcolor[HTML]{8FD2B1}0.0435                 & \cellcolor[HTML]{8FD2B1}0.0435                 & \cellcolor[HTML]{C7E9D8}0.0217                 & \cellcolor[HTML]{C7E9D8}0.0217                 & \cellcolor[HTML]{C7E9D8}0.0217                 & \cellcolor[HTML]{C7E9D8}0.0217                  & \cellcolor[HTML]{FFFFFF}0                       & \cellcolor[HTML]{FFFFFF}0                       & \cellcolor[HTML]{FFFFFF}0                       & \cellcolor[HTML]{FFFFFF}0                       & \cellcolor[HTML]{D9D2E9}0.2391                                                  & \cellcolor[HTML]{D9D2E9}82.5                                            & \cellcolor[HTML]{D9D2E9}19.72826087                                                              \\ \hline
            \cellcolor[HTML]{C8E4BE}85             & \cellcolor[HTML]{D9EAD3}90             & \cellcolor[HTML]{C7E9D8}0.0217                 & \cellcolor[HTML]{C7E9D8}0.0217                 & \cellcolor[HTML]{8FD2B1}0.0435                 & \cellcolor[HTML]{57BB8A}0.0652                 & \cellcolor[HTML]{8FD2B1}0.0435                 & \cellcolor[HTML]{8FD2B1}0.0435                 & \cellcolor[HTML]{8FD2B1}0.0435                 & \cellcolor[HTML]{C7E9D8}0.0217                  & \cellcolor[HTML]{C7E9D8}0.0217                  & \cellcolor[HTML]{FFFFFF}0                       & \cellcolor[HTML]{FFFFFF}0                       & \cellcolor[HTML]{FFFFFF}0                       & \cellcolor[HTML]{D9D2E9}0.3261                                                  & \cellcolor[HTML]{D9D2E9}87.5                                            & \cellcolor[HTML]{D9D2E9}28.5326                                                                  \\ \hline
            \rowcolor[HTML]{FFFFFF} 
            \cellcolor[HTML]{C8E4BE}90             & \cellcolor[HTML]{D9EAD3}95             & 0                                              & 0                                              & 0                                              & \cellcolor[HTML]{8FD2B1}0.0435                 & 0                                              & 0                                              & \cellcolor[HTML]{C7E9D8}0.0217                 & 0                                               & 0                                               & 0                                               & 0                                               & 0                                               & \cellcolor[HTML]{D9D2E9}0.0652                                                  & \cellcolor[HTML]{D9D2E9}92.5                                            & \cellcolor[HTML]{D9D2E9}6.0326                                                                   \\ \hline
            \rowcolor[HTML]{FFFFFF} 
            \cellcolor[HTML]{C8E4BE}95             & \cellcolor[HTML]{D9EAD3}100            & \cellcolor[HTML]{C7E9D8}0.0217                 & 0                                              & 0                                              & 0                                              & 0                                              & 0                                              & 0                                              & 0                                               & 0                                               & 0                                               & 0                                               & \cellcolor[HTML]{C7E9D8}0.0217                  & \cellcolor[HTML]{D9D2E9}0.0435                                                  & \cellcolor[HTML]{D9D2E9}97.5                                            & \cellcolor[HTML]{D9D2E9}4.2391                                                                   \\ \hline
            \rowcolor[HTML]{FFFFFF} 
            \cellcolor[HTML]{C8E4BE}100            & \cellcolor[HTML]{D9EAD3}105            & 0                                              & 0                                              & \cellcolor[HTML]{C7E9D8}0.0217                 & 0                                              & 0                                              & 0                                              & 0                                              & 0                                               & 0                                               & 0                                               & 0                                               & 0                                               & \cellcolor[HTML]{D9D2E9}0.0217                                                  & \cellcolor[HTML]{D9D2E9}102.5                                           & \cellcolor[HTML]{D9D2E9}2.2283                                                                   \\ \hline
            \multicolumn{2}{|c|}{\cellcolor[HTML]{FCE5CD}$\widehat{P}(X)$}                  & \cellcolor[HTML]{FCE5CD}0.0652                 & \cellcolor[HTML]{FCE5CD}0.1522                 & \cellcolor[HTML]{FCE5CD}0.1304                 & \cellcolor[HTML]{FCE5CD}0.1739                 & \cellcolor[HTML]{FCE5CD}0.087                  & \cellcolor[HTML]{FCE5CD}0.1304                 & \cellcolor[HTML]{FCE5CD}0.1304                 & \cellcolor[HTML]{FCE5CD}0.087                   & \cellcolor[HTML]{FCE5CD}0.0217                  & \cellcolor[HTML]{FCE5CD}0                       & \cellcolor[HTML]{FCE5CD}0                       & \cellcolor[HTML]{FCE5CD}0.0217                  & \multicolumn{1}{l|}{}                                                           & \multicolumn{1}{l|}{}                                                   & \multicolumn{1}{l|}{}                                                                            \\ \hline
            \multicolumn{2}{|c|}{\cellcolor[HTML]{FCE5CD}midpoint}                          & \cellcolor[HTML]{FCE5CD}2.5                    & \cellcolor[HTML]{FCE5CD}3.5                    & \cellcolor[HTML]{FCE5CD}4.5                    & \cellcolor[HTML]{FCE5CD}5.5                    & \cellcolor[HTML]{FCE5CD}6.5                    & \cellcolor[HTML]{FCE5CD}7.5                    & \cellcolor[HTML]{FCE5CD}8.5                    & \cellcolor[HTML]{FCE5CD}9.5                     & \cellcolor[HTML]{FCE5CD}10.5                    & \cellcolor[HTML]{FCE5CD}11.5                    & \cellcolor[HTML]{FCE5CD}12.5                    & \cellcolor[HTML]{FCE5CD}13.5                    & \multicolumn{1}{l|}{}                                                           & \multicolumn{1}{l|}{\cellcolor[HTML]{D9D9D9}$\widehat{E}(X)$}           & \cellcolor[HTML]{D9D9D9}6.2391                                                                   \\ \hline
            \multicolumn{2}{|c|}{\cellcolor[HTML]{FCE5CD}$\mathrm{mid}\cdot\widehat{P}(X)$} & \cellcolor[HTML]{FCE5CD}0.163                  & \cellcolor[HTML]{FCE5CD}0.5326                 & \cellcolor[HTML]{FCE5CD}0.587                  & \cellcolor[HTML]{FCE5CD}0.9565                 & \cellcolor[HTML]{FCE5CD}0.5652                 & \cellcolor[HTML]{FCE5CD}0.9783                 & \cellcolor[HTML]{FCE5CD}1.1087                 & \cellcolor[HTML]{FCE5CD}0.8261                  & \cellcolor[HTML]{FCE5CD}0.2283                  & \cellcolor[HTML]{FCE5CD}0                       & \cellcolor[HTML]{FCE5CD}0                       & \cellcolor[HTML]{FCE5CD}0.2935                  & \multicolumn{1}{l|}{}                                                           & \multicolumn{1}{l|}{\cellcolor[HTML]{D9D9D9}$\widehat{E}(Y)$}           & \cellcolor[HTML]{D9D9D9}81.5217                                                                  \\ \hline
        \end{tabular}
    \end{adjustbox}
\end{table}\\
\noindent The individual probability from the table comes from dividing frequency with the number of samples. $P[X]$ comes from summing the data for all columns. $P[Y]$ comes from summing the data for all rows. The expected value of $X$ $(E[X])$ is 6.23913 which is calculated by summing all of $P[X]$ and the expected value of $Y$ $(E[Y])$ is 81.5217 which is calculated by summing all of $P[Y]$.\\
\begin{figure}[ht!]
    \begin{tikzpicture}[scale=0.51]
        \begin{axis}
            [
                xlabel = screen duration (Hrs),
                ylabel = \(\widehat{P}(X)\),
                width = 0.9\textwidth,
                grid,
                ybar,
                ymin = 0,
                xmin = 2, xmax = 14,
                xtick distance = 1,
                % ytick distance = 0.05,
                title = \(\widehat{P}(X)\),
                % nodes near coords style={font=\tiny}
            ]
            \addplot+[ybar interval] table [y=Px_SA, col sep=comma, x expr=\coordindex+2] {6 graph.csv};
        \end{axis}
    \end{tikzpicture}
    \begin{tikzpicture}[scale=0.51]
        \begin{axis}
            [
                xlabel = Heart rate,
                ylabel = \(\widehat{P}(Y)\),
                width = 0.9\textwidth,
                grid,
                ybar,
                ymin = 0,
                xmin = 0, xmax = 11 ,
                xtick = data,
                xticklabels = {50, 55, 60, 65, 70, 75, 80, 85, 90, 95, 100, 105, 110, 115},
                % xtick distance =2,
                % ytick distance = 0.025,
                title = \(\widehat{P}(Y)\),
                % nodes near coords style={font=\tiny}
            ]
            \addplot+[ybar interval] table [y=Py_SH, col sep=comma, x expr=\coordindex] {6 graph.csv};
        \end{axis}
    \end{tikzpicture}
\end{figure}\\
The correlation coefficient is computed by using the formula
\begin{displaymath}
    \widehat{\rho}_{X, Y} = \dfrac{\widehat{S}_{X, Y}}{s_Xs_Y} \enskip \& \enskip \widehat{S}_{X, Y}=\dfrac{1}{n-1}\dsl^n_{i=1}\left(x_i-\overline{X}\right)\left(y_i-\overline{Y}\right)
\end{displaymath}
In this case, \(\overline{X} = 6.271, \overline{Y} = 80.7174, n = 46, s_x = 2.4862, s_y = 10.8805\). Therefore, \(\widehat{S}_{X, Y} = -1.3246\) and \(\widehat{\rho}_{X, Y} = -0.0490\). Thus, the relation between $X$ and $Y$ has a weak negative relationship.\newpage
\subsection*{Screen time vs Weight}
\begin{table}[ht]
    \begin{adjustbox}{width = \textwidth, center}
        \begin{tabular}{|cc|r|r|r|r|r|r|r|r|r|r|r|r|r|r|r|}
            \hline
            \multicolumn{2}{|c|}{}                                                          & \multicolumn{1}{c|}{\cellcolor[HTML]{F4CCCC}2} & \multicolumn{1}{c|}{\cellcolor[HTML]{F4CCCC}3} & \multicolumn{1}{c|}{\cellcolor[HTML]{F4CCCC}4} & \multicolumn{1}{c|}{\cellcolor[HTML]{F4CCCC}5} & \multicolumn{1}{c|}{\cellcolor[HTML]{F4CCCC}6} & \multicolumn{1}{c|}{\cellcolor[HTML]{F4CCCC}7} & \multicolumn{1}{c|}{\cellcolor[HTML]{F4CCCC}8} & \multicolumn{1}{c|}{\cellcolor[HTML]{F4CCCC}9}  & \multicolumn{1}{c|}{\cellcolor[HTML]{F4CCCC}10} & \multicolumn{1}{c|}{\cellcolor[HTML]{F4CCCC}11} & \multicolumn{1}{c|}{\cellcolor[HTML]{F4CCCC}12} & \multicolumn{1}{c|}{\cellcolor[HTML]{F4CCCC}13} & \multicolumn{1}{c|}{\cellcolor[HTML]{D9D2E9}}                                   & \multicolumn{1}{c|}{\cellcolor[HTML]{D9D2E9}}                           & \multicolumn{1}{c|}{\cellcolor[HTML]{D9D2E9}}                                                    \\
            \multicolumn{2}{|c|}{\multirow{-2}{*}{\backslashbox{$y$}{$x$}}}                 & \multicolumn{1}{c|}{\cellcolor[HTML]{FFEBEA}3} & \multicolumn{1}{c|}{\cellcolor[HTML]{FFEBEA}4} & \multicolumn{1}{c|}{\cellcolor[HTML]{FFEBEA}5} & \multicolumn{1}{c|}{\cellcolor[HTML]{FFEBEA}6} & \multicolumn{1}{c|}{\cellcolor[HTML]{FFEBEA}7} & \multicolumn{1}{c|}{\cellcolor[HTML]{FFEBEA}8} & \multicolumn{1}{c|}{\cellcolor[HTML]{FFEBEA}9} & \multicolumn{1}{c|}{\cellcolor[HTML]{FFEBEA}10} & \multicolumn{1}{c|}{\cellcolor[HTML]{FFEBEA}11} & \multicolumn{1}{c|}{\cellcolor[HTML]{FFEBEA}12} & \multicolumn{1}{c|}{\cellcolor[HTML]{FFEBEA}13} & \multicolumn{1}{c|}{\cellcolor[HTML]{FFEBEA}14} & \multicolumn{1}{c|}{\multirow{-2}{*}{\cellcolor[HTML]{D9D2E9}$\widehat{P}(Y)$}} & \multicolumn{1}{c|}{\multirow{-2}{*}{\cellcolor[HTML]{D9D2E9}midpoint}} & \multicolumn{1}{c|}{\multirow{-2}{*}{\cellcolor[HTML]{D9D2E9}$\mathrm{mid}\cdot\widehat{P}(Y)$}} \\ \hline
            \rowcolor[HTML]{FFFFFF} 
            \cellcolor[HTML]{C9DAF8}\hspace{6pt}25\hspace{6pt}             & \cellcolor[HTML]{EBF1FC}35             & 0                                              & 0                                              & 0                                              & 0                                              & 0                                              & 0                                              & 0                                              & 0                                               & 0                                               & 0                                               & 0                                               & 0                                               & \cellcolor[HTML]{D9D2E9}0                                                       & \cellcolor[HTML]{D9D2E9}30                                              & \cellcolor[HTML]{D9D2E9}0                                                                        \\ \hline
            \rowcolor[HTML]{FFFFFF} 
            \cellcolor[HTML]{C9DAF8}35             & \cellcolor[HTML]{EBF1FC}45             & 0                                              & 0                                              & 0                                              & 0                                              & 0                                              & 0                                              & \cellcolor[HTML]{DEF2E8}0.0217                 & 0                                               & 0                                               & 0                                               & 0                                               & 0                                               & \cellcolor[HTML]{D9D2E9}0.0217                                                  & \cellcolor[HTML]{D9D2E9}40                                              & \cellcolor[HTML]{D9D2E9}0.8696                                                                   \\ \hline
            \cellcolor[HTML]{C9DAF8}45             & \cellcolor[HTML]{EBF1FC}55             & \cellcolor[HTML]{FFFFFF}0                      & \cellcolor[HTML]{57BB8A}0.1087                 & \cellcolor[HTML]{FFFFFF}0                      & \cellcolor[HTML]{9BD7B9}0.0652                 & \cellcolor[HTML]{FFFFFF}0                      & \cellcolor[HTML]{BCE4D1}0.0435                 & \cellcolor[HTML]{DEF2E8}0.0217                 & \cellcolor[HTML]{DEF2E8}0.0217                  & \cellcolor[HTML]{FFFFFF}0                       & \cellcolor[HTML]{FFFFFF}0                       & \cellcolor[HTML]{FFFFFF}0                       & \cellcolor[HTML]{FFFFFF}0                       & \cellcolor[HTML]{D9D2E9}0.2609                                                  & \cellcolor[HTML]{D9D2E9}50                                              & \cellcolor[HTML]{D9D2E9}13.0435                                                                  \\ \hline
            \cellcolor[HTML]{C9DAF8}55             & \cellcolor[HTML]{EBF1FC}65             & \cellcolor[HTML]{BCE4D1}0.0435                 & \cellcolor[HTML]{DEF2E8}0.0217                 & \cellcolor[HTML]{9BD7B9}0.0652                 & \cellcolor[HTML]{79C9A2}0.087                  & \cellcolor[HTML]{FFFFFF}0                      & \cellcolor[HTML]{BCE4D1}0.0435                 & \cellcolor[HTML]{9BD7B9}0.0652                 & \cellcolor[HTML]{9BD7B9}0.0652                  & \cellcolor[HTML]{DEF2E8}0.0217                  & \cellcolor[HTML]{FFFFFF}0                       & \cellcolor[HTML]{FFFFFF}0                       & \cellcolor[HTML]{DEF2E8}0.0217                  & \cellcolor[HTML]{D9D2E9}0.4348                                                  & \cellcolor[HTML]{D9D2E9}60                                              & \cellcolor[HTML]{D9D2E9}26.087                                                                   \\ \hline
            \cellcolor[HTML]{C9DAF8}65             & \cellcolor[HTML]{EBF1FC}75             & \cellcolor[HTML]{DEF2E8}0.0217                 & \cellcolor[HTML]{FFFFFF}0                      & \cellcolor[HTML]{BCE4D1}0.0435                 & \cellcolor[HTML]{DEF2E8}0.0217                 & \cellcolor[HTML]{BCE4D1}0.0435                 & \cellcolor[HTML]{DEF2E8}0.0217                 & \cellcolor[HTML]{FFFFFF}0                      & \cellcolor[HTML]{FFFFFF}0                       & \cellcolor[HTML]{FFFFFF}0                       & \cellcolor[HTML]{FFFFFF}0                       & \cellcolor[HTML]{FFFFFF}0                       & \cellcolor[HTML]{FFFFFF}0                       & \cellcolor[HTML]{D9D2E9}0.1522                                                  & \cellcolor[HTML]{D9D2E9}70                                              & \cellcolor[HTML]{D9D2E9}10.6522                                                                  \\ \hline
            \cellcolor[HTML]{C9DAF8}75             & \cellcolor[HTML]{EBF1FC}85             & \cellcolor[HTML]{FFFFFF}0                      & \cellcolor[HTML]{FFFFFF}0                      & \cellcolor[HTML]{DEF2E8}0.0217                 & \cellcolor[HTML]{FFFFFF}0                      & \cellcolor[HTML]{BCE4D1}0.0435                 & \cellcolor[HTML]{DEF2E8}0.0217                 & \cellcolor[HTML]{DEF2E8}0.0217                 & \cellcolor[HTML]{FFFFFF}0                       & \cellcolor[HTML]{FFFFFF}0                       & \cellcolor[HTML]{FFFFFF}0                       & \cellcolor[HTML]{FFFFFF}0                       & \cellcolor[HTML]{FFFFFF}0                       & \cellcolor[HTML]{D9D2E9}0.1087                                                  & \cellcolor[HTML]{D9D2E9}80                                              & \cellcolor[HTML]{D9D2E9}8.6957                                                                   \\ \hline
            \rowcolor[HTML]{FFFFFF} 
            \cellcolor[HTML]{C9DAF8}85             & \cellcolor[HTML]{EBF1FC}95             & 0                                              & \cellcolor[HTML]{DEF2E8}0.0217                 & 0                                              & 0                                              & 0                                              & 0                                              & 0                                              & 0                                               & 0                                               & 0                                               & 0                                               & 0                                               & \cellcolor[HTML]{D9D2E9}0.0217                                                  & \cellcolor[HTML]{D9D2E9}90                                              & \cellcolor[HTML]{D9D2E9}1.9565                                                                   \\ \hline
            \multicolumn{2}{|c|}{\cellcolor[HTML]{FCE5CD}$\widehat{P}(X)$}                  & \cellcolor[HTML]{FCE5CD}0.0652                 & \cellcolor[HTML]{FCE5CD}0.1522                 & \cellcolor[HTML]{FCE5CD}0.1304                 & \cellcolor[HTML]{FCE5CD}0.1739                 & \cellcolor[HTML]{FCE5CD}0.087                  & \cellcolor[HTML]{FCE5CD}0.1304                 & \cellcolor[HTML]{FCE5CD}0.1304                 & \cellcolor[HTML]{FCE5CD}0.087                   & \cellcolor[HTML]{FCE5CD}0.0217                  & \cellcolor[HTML]{FCE5CD}0                       & \cellcolor[HTML]{FCE5CD}0                       & \cellcolor[HTML]{FCE5CD}0.0217                  & \multicolumn{1}{l|}{}                                                           & \multicolumn{1}{l|}{}                                                   & \multicolumn{1}{l|}{}                                                                            \\ \hline
            \multicolumn{2}{|c|}{\cellcolor[HTML]{FCE5CD}midpoint}                          & \cellcolor[HTML]{FCE5CD}2.5                    & \cellcolor[HTML]{FCE5CD}3.5                    & \cellcolor[HTML]{FCE5CD}4.5                    & \cellcolor[HTML]{FCE5CD}5.5                    & \cellcolor[HTML]{FCE5CD}6.5                    & \cellcolor[HTML]{FCE5CD}7.5                    & \cellcolor[HTML]{FCE5CD}8.5                    & \cellcolor[HTML]{FCE5CD}9.5                     & \cellcolor[HTML]{FCE5CD}10.5                    & \cellcolor[HTML]{FCE5CD}11.5                    & \cellcolor[HTML]{FCE5CD}12.5                    & \cellcolor[HTML]{FCE5CD}13.5                    & \multicolumn{1}{l|}{}                                                           & \multicolumn{1}{c|}{\cellcolor[HTML]{C9DAF8}$\widehat{E}(X)$}           & \cellcolor[HTML]{C9DAF8}6.2391                                                                   \\ \hline
            \multicolumn{2}{|c|}{\cellcolor[HTML]{FCE5CD}$\mathrm{mid}\cdot\widehat{P}(X)$} & \cellcolor[HTML]{FCE5CD}0.163                  & \cellcolor[HTML]{FCE5CD}0.5326                 & \cellcolor[HTML]{FCE5CD}0.587                  & \cellcolor[HTML]{FCE5CD}0.9565                 & \cellcolor[HTML]{FCE5CD}0.5652                 & \cellcolor[HTML]{FCE5CD}0.9783                 & \cellcolor[HTML]{FCE5CD}1.1087                 & \cellcolor[HTML]{FCE5CD}0.8261                  & \cellcolor[HTML]{FCE5CD}0.2283                  & \cellcolor[HTML]{FCE5CD}0                       & \cellcolor[HTML]{FCE5CD}0                       & \cellcolor[HTML]{FCE5CD}0.2935                  & \multicolumn{1}{l|}{}                                                           & \multicolumn{1}{c|}{\cellcolor[HTML]{C9DAF8}$\widehat{E}(Y)$}           & \cellcolor[HTML]{C9DAF8}61.3043                                                                  \\ \hline
        \end{tabular}
    \end{adjustbox}
\end{table}\\
\noindent The individual probability from the table comes from dividing frequency with the number of samples. $P[X]$ comes from summing the data for all columns. $P[Y]$ comes from summing the data for all rows. The expected value of $X$ $(E[X])$ is 6.23913 which is calculated by summing all of $P[X]$ and the expected value of $Y$ $(E[Y])$ is 61.3043 which is calculated by summing all of $P[Y]$.\\
\begin{figure}[ht!]
    \begin{tikzpicture}[scale=0.51]
        \begin{axis}
            [
                xlabel = screen duration (Hrs),
                ylabel = \(\widehat{P}(X)\),
                width = 0.9\textwidth,
                grid,
                ybar,
                ymin = 0,
                xmin = 2, xmax = 14,
                xtick distance = 1,
                % ytick distance = 0.05,
                title = \(\widehat{P}(X)\),
                % nodes near coords style={font=\tiny}
            ]
            \addplot+[ybar interval] table [y=Px_SA, col sep=comma, x expr=\coordindex+2] {6 graph.csv};
        \end{axis}
    \end{tikzpicture}
    \begin{tikzpicture}[scale=0.51]
        \begin{axis}
            [
                xlabel = Weight,
                ylabel = \(\widehat{P}(Y)\),
                width = 0.9\textwidth,
                grid,
                ybar,
                ymin = 0,
                xmin = 0, xmax = 7,
                % xtick distance =2,
                xtick = data,
                xticklabels = {25, 35, 45, 55, 65, 75, 85, 95},
                % ytick distance = 0.025,
                title = \(\widehat{P}(Y)\),
                % nodes near coords style={font=\tiny}
            ]
            \addplot+[ybar interval] table [y=Py_SW, col sep=comma, x expr=\coordindex] {6 graph.csv};
        \end{axis}
    \end{tikzpicture}
\end{figure}\\
The correlation coefficient is computed by using the formula
\begin{displaymath}
    \widehat{\rho}_{X, Y} = \dfrac{\widehat{S}_{X, Y}}{s_Xs_Y} \enskip \& \enskip \widehat{S}_{X, Y}=\dfrac{1}{n-1}\dsl^n_{i=1}\left(x_i-\overline{X}\right)\left(y_i-\overline{Y}\right)
\end{displaymath}
In this case, \(\overline{X} = 6.271, \overline{Y} = 61.0587, n = 46, s_x = 2.4862, s_y = 10.3099\). Therefore, \(\widehat{S}_{X, Y} = -0.3518\) and \(\widehat{\rho}_{X, Y} = -0.0137\). Thus, the relation between $X$ and $Y$ has a weak negative relationship.\\
\section*{Statistics}
\subsection*{Descriptive Statistics}\\
\begin{table}[H]
    \begin{adjustbox}{width = \textwidth, center}
        \begin{tabular}{|c|rrrrrr|}
            \hline
            \multicolumn{1}{|l|}{}                                                       & \multicolumn{1}{c|}{\cellcolor[HTML]{FFE599}Age}                & \multicolumn{1}{c|}{\cellcolor[HTML]{FFE599}Average daily screen time (Hrs)} & \multicolumn{1}{c|}{\cellcolor[HTML]{FFE599}Average daily notification received} & \multicolumn{1}{c|}{\cellcolor[HTML]{FFE599}Weight (Kg)} & \multicolumn{1}{c|}{\cellcolor[HTML]{FFE599}Heart Rate} & \multicolumn{1}{c|}{\cellcolor[HTML]{FFE599}Sleeping time (Hrs)} \\ \hline
            \cellcolor[HTML]{FFFF00}Count                                                & \multicolumn{1}{r|}{55}                                         & \multicolumn{1}{r|}{55}                                                      & \multicolumn{1}{r|}{55}                                                          & \multicolumn{1}{r|}{55}                                  & \multicolumn{1}{r|}{55}                                 & 55                                                               \\ \hline
            \cellcolor[HTML]{B6D7A8}MEAN                                                 & \multicolumn{1}{r|}{29.1818}                                    & \multicolumn{1}{r|}{6.5771}                                                  & \multicolumn{1}{r|}{103.7506}                                                    & \multicolumn{1}{r|}{61.9709}                             & \multicolumn{1}{r|}{80.8909}                            & 7.3098                                                           \\ \hline
            \cellcolor[HTML]{B6D7A8}median                                               & \multicolumn{1}{r|}{21}                                         & \multicolumn{1}{r|}{6.2024}                                                  & \multicolumn{1}{r|}{86}                                                          & \multicolumn{1}{r|}{60.000}                              & \multicolumn{1}{r|}{83}                                 & 7.34                                                             \\ \hline
            \cellcolor[HTML]{B6D7A8}mode                                                 & \multicolumn{1}{r|}{19}                                         & \multicolumn{1}{r|}{6.2024}                                                  & \multicolumn{1}{r|}{86}                                                          & \multicolumn{1}{r|}{50.000}                              & \multicolumn{1}{r|}{87}                                 & 6.50                                                             \\ \hline
            \cellcolor[HTML]{CFE2F3}MIN                                                  & \multicolumn{1}{r|}{7}                                          & \multicolumn{1}{r|}{2.2357}                                                  & \multicolumn{1}{r|}{6.4286}                                                      & \multicolumn{1}{r|}{25.70}                               & \multicolumn{1}{r|}{51}                                 & 4.50                                                             \\ \hline
            \cellcolor[HTML]{CFE2F3}MAX                                                  & \multicolumn{1}{r|}{57}                                         & \multicolumn{1}{r|}{16.6714}                                                 & \multicolumn{1}{r|}{357}                                                         & \multicolumn{1}{r|}{111.1}                               & \multicolumn{1}{r|}{111}                                & 10.5333                                                          \\ \hline
            \cellcolor[HTML]{CFE2F3}range                                                & \multicolumn{1}{r|}{50}                                         & \multicolumn{1}{r|}{14.4357}                                                 & \multicolumn{1}{r|}{350.5714}                                                    & \multicolumn{1}{r|}{85.4}                                & \multicolumn{1}{r|}{60}                                 & 6.0333                                                           \\ \hline
            \cellcolor[HTML]{CFE2F3}variance                                             & \multicolumn{1}{r|}{175.4478}                                   & \multicolumn{1}{r|}{7.9495}                                                  & \multicolumn{1}{r|}{5585.4741}                                                   & \multicolumn{1}{r|}{194.314}                             & \multicolumn{1}{r|}{168.4323}                           & 1.4176                                                           \\ \hline
            \cellcolor[HTML]{CFE2F3}SD                                                   & \multicolumn{1}{r|}{13.2457}                                    & \multicolumn{1}{r|}{2.8195}                                                  & \multicolumn{1}{r|}{74.736}                                                      & \multicolumn{1}{r|}{13.9397}                             & \multicolumn{1}{r|}{12.9781}                            & 1.1906                                                           \\ \hline
            \cellcolor[HTML]{CFE2F3}cv                                                   & \multicolumn{1}{r|}{0.4539}                                     & \multicolumn{1}{r|}{0.4287}                                                  & \multicolumn{1}{r|}{0.7203}                                                      & \multicolumn{1}{r|}{0.2249}                              & \multicolumn{1}{r|}{0.1604}                             & 0.1629                                                           \\ \hline
            \cellcolor[HTML]{CFE2F3}MAD                                                  & \multicolumn{1}{r|}{11.7752}                                    & \multicolumn{1}{r|}{2.2193}                                                  & \multicolumn{1}{r|}{56.0073}                                                     & \multicolumn{1}{r|}{9.9298}                              & \multicolumn{1}{r|}{9.9451}                             & 0.8687                                                           \\ \hline
            quartile1 (Q1)                                                               & \multicolumn{1}{r|}{19}                                         & \multicolumn{1}{r|}{4.569}                                                   & \multicolumn{1}{r|}{49.1429}                                                     & \multicolumn{1}{r|}{52.5}                                & \multicolumn{1}{r|}{73}                                 & 6.5                                                              \\ \hline
            quartile3 (Q3)                                                               & \multicolumn{1}{r|}{43}                                         & \multicolumn{1}{r|}{8.3167}                                                  & \multicolumn{1}{r|}{146.7143}                                                    & \multicolumn{1}{r|}{67}                                  & \multicolumn{1}{r|}{88}                                 & 8                                                                \\ \hline
            \cellcolor[HTML]{CFE2F3}IQR                                                  & \multicolumn{1}{r|}{24}                                         & \multicolumn{1}{r|}{3.7476}                                                  & \multicolumn{1}{r|}{97.5714}                                                     & \multicolumn{1}{r|}{14.5}                                & \multicolumn{1}{r|}{15}                                 & 1.5                                                              \\ \hline
            Q1-1.5IQR                                                                    & \multicolumn{1}{r|}{-17}                                        & \multicolumn{1}{r|}{-1.0524}                                                 & \multicolumn{1}{r|}{-97.2143}                                                    & \multicolumn{1}{r|}{30.75}                               & \multicolumn{1}{r|}{50.5}                               & 4.25                                                             \\ \hline
            Q3+1.5IQR                                                                    & \multicolumn{1}{r|}{79}                                         & \multicolumn{1}{r|}{13.9381}                                                 & \multicolumn{1}{r|}{293.0714}                                                    & \multicolumn{1}{r|}{88.75}                               & \multicolumn{1}{r|}{110.5}                              & 10.25                                                            \\ \hline
            \cellcolor[HTML]{D9D2E9}Outliers (based on IQR)  If no outlier, answer None. & \multicolumn{1}{r|}{None}                                       & \multicolumn{1}{r|}{16.671}                                                  & \multicolumn{1}{r|}{294.857, 357}                                                & \multicolumn{1}{r|}{25.7, 93, 111.1}                     & \multicolumn{1}{r|}{111}                                & 10.417, 10.533                                                   \\ \hline
            MEAN-3SD                                                                     & \multicolumn{1}{r|}{-10.5552}                                   & \multicolumn{1}{r|}{-1.8814}                                                 & \multicolumn{1}{r|}{-120.4574}                                                   & \multicolumn{1}{r|}{20.1519}                             & \multicolumn{1}{r|}{41.9565}                            & 3.7379                                                           \\ \hline
            MEAN+3SD                                                                     & \multicolumn{1}{r|}{68.9188}                                    & \multicolumn{1}{r|}{15.0355}                                                 & \multicolumn{1}{r|}{327.9587}                                                    & \multicolumn{1}{r|}{103.7899}                            & \multicolumn{1}{r|}{119.8254}                           & 10.8818                                                          \\ \hline
            \cellcolor[HTML]{D9D2E9}Outliers (based on SD)  If no outlier, answer None.  & \multicolumn{1}{r|}{None}                                       & \multicolumn{1}{r|}{16.6714}                                                 & \multicolumn{1}{r|}{357}                                                         & \multicolumn{1}{r|}{111.1}                               & \multicolumn{1}{r|}{None}                               & None                                                             \\ \hline
            Mean after removing outliers based on IQR. If no outlier, type NA            & \multicolumn{1}{r|}{NA}                                         & \multicolumn{1}{r|}{6.3901}                                                  & \multicolumn{1}{r|}{95.3666}                                                     & \multicolumn{1}{r|}{61.1269}                             & \multicolumn{1}{r|}{80.3333}                            & 7.1904                                                           \\ \hline
            SD after removing outliers based on IQR. If no outlier, type NA              & \multicolumn{1}{r|}{NA}                                         & \multicolumn{1}{r|}{2.4782}                                                  & \multicolumn{1}{r|}{61.5869}                                                     & \multicolumn{1}{r|}{10.6309}                             & \multicolumn{1}{r|}{12.4173}                            & 1.0355                                                           \\ \hline
            \rowcolor[HTML]{F4CCCC} 
            Measure of Centrality                                                        & \multicolumn{6}{c|}{\cellcolor[HTML]{F4CCCC}Median}                                                                                                                                                                                                                                                                                                                                                                       \\ \hline
            \rowcolor[HTML]{F4CCCC} 
            Reason                                                                       & \multicolumn{4}{c|}{\cellcolor[HTML]{F4CCCC}The data is right-skewed}                                                                                                                                                                                                                        & \multicolumn{2}{c|}{\cellcolor[HTML]{F4CCCC}The data is left-skewed}                                                       \\ \hline
            \rowcolor[HTML]{FCE5CD} 
            Measure of Dispersion                                                         & \multicolumn{6}{c|}{\cellcolor[HTML]{FCE5CD}IQR}                                                                                                                                                                                                                                                                                                                                                                          \\ \hline
            \rowcolor[HTML]{FCE5CD} 
            Reason                                                                       & \multicolumn{1}{c|}{\cellcolor[HTML]{FCE5CD}The data is skewed} & \multicolumn{5}{c|}{\cellcolor[HTML]{FCE5CD}The dataset has an outlier}                                                                                                                                                                                                                                                                                 \\ \hline
            \end{tabular}
    \end{adjustbox}
\end{table}\\
\subsection*{Goodness of Fit test}

\begin{enumerate}
    \item \textbf{Data set : Average screen duration}\\*[10pt]
        Type of distribution : Normal distribution\\*[5pt]
        Known parameter : 0\\*[5pt]
        Unknown parameter ($m$) : 2, which are \(\mu, \sigma\)\\*[5pt]
        $H_0$ : Screen duration is normally distributed with $\mu=6.2709$ and $\sigma=2.4862$\\*[5pt]
        $H_a$ : Screen duration is not normally distributed with $\mu=6.2709$ and $\sigma=2.4862$\\*[5pt]
        Number of cells with the expected number of samples ($k$) : 6\\*[5pt]
        Test static \(\chi^2=\displaystyle\sum\limits^k_{i=1}\dfrac{\left(O_i-E_i\right)^2}{E_i} = 2.0956\)\\*[5pt]
        Significant level \(\left(\alpha\right)\) : 0.05\\*[5pt]
        Degree of freedom 1 \((\nu_1)\) : $k - 1 - m = 6 - 1 - 2 \Rightarrow 3$\\*[5pt]
        Cutoff of non-rejection region : 7.8147\\*[5pt]
        Degree of freedom 2 \((\nu_2)\) : $k - 1 = 6 - 1 \Rightarrow 5$\\*[5pt]
        Cutoff of rejection region : 11.070\\*[5pt]
        Non-rejection regions : \(\chi^2 < \chi^2_{0.05, 3}=7.8147\)\\*[5pt]
        Rejection regions : \(\chi^2 \geq \chi^2_{0.05, 5}=11.070\)\\*[5pt]
        Rejection decision : Don't need to reject null hypothesis\\*[5pt]
        Conclusion : Screen duration is normally distributed with $\mu=6.2709$ and $\sigma=2.4862$\\*[5pt]
    \item \textbf{Data set : Average Sleeping Duration}\\*[10pt]
        Type of distribution : Normal distribution\\*[5pt]
        Known parameter : 0\\*[5pt]
        Unknown parameter ($m$) : 2, which are \(\mu, \sigma\)\\*[5pt]
        $H_0$ : Screen duration is normally distributed with $\mu=7.2178$ and $\sigma=1.0068$\\*[5pt]
        $H_a$ : Screen duration is not normally distributed with $\mu=7.2178$ and $\sigma=1.0068$\\*[5pt]
        Number of cells with the expected number of samples ($k$) : 4\\*[5pt]
        Test static \(\chi^2=\displaystyle\sum\limits^k_{i=1}\dfrac{\left(O_i-E_i\right)^2}{E_i}=1.3428\)\\*[5pt]
        Significant level \(\left(\alpha\right)\) : 0.05\\*[5pt]
        Degree of freedom 1 \((\nu_1)\) : $k - 1 - m = 4 - 1 - 2 \Rightarrow 1$\\*[5pt]
        Cutoff of non-rejection region : 3.8415\\*[5pt]
        Degree of freedom 2 \((\nu_2)\) : $k - 1 = 4 - 1 \Rightarrow 3$\\*[5pt]
        Cutoff of rejection region : 7.8147\\*[5pt]
        Non-rejection regions : \(\chi^2 < \chi^2_{0.05, 1}=3.8415\)\\*[5pt]
        Rejection regions : \(\chi^2 \geq \chi^2_{0.05, 3}=7.8147\)\\*[5pt]
        Rejection decision : Don't need to reject null hypothesis\\*[5pt]
        Conclusion : Screen duration is normally distributed with $\mu=7.2178$ and $\sigma=1.0068$\\*[5pt]
    \item \textbf{Data set : Weight}\\*[10pt]
        Type of distribution : Normal distribution\\*[5pt]
        Known parameter : 0\\*[5pt]
        Unknown parameter ($m$) : 2, which are \(\mu, \sigma\)\\*[5pt]
        $H_0$ : Screen duration is normally distributed with $\mu=61.0587$ and $\sigma=10.3099$\\*[5pt]
        $H_a$ : Screen duration is not normally distributed with $\mu=61.0587$ and $\sigma=10.3099$\\*[5pt]
        Number of cells with the expected number of samples ($k$) : 6\\*[5pt]
        Test static \(\chi^2=\displaystyle\sum\limits^k_{i=1}\dfrac{\left(O_i-E_i\right)^2}{E_i}=3.1761\)\\*[5pt]
        Significant level \(\left(\alpha\right)\) : 0.05\\*[5pt]
        Degree of freedom 1 \((\nu_1)\) : $k - 1 - m = 6 - 1 - 2 \Rightarrow 3$\\*[5pt]
        Cutoff of non-rejection region : 7.8147\\*[5pt]
        Degree of freedom 2 \((\nu_2)\) : $k - 1 = 6 - 1 \Rightarrow 5$\\*[5pt]
        Cutoff of rejection region : 11.070\\*[5pt]
        Non-rejection regions : \(\chi^2 < \chi^2_{0.05, 3}=7.8147\)\\*[5pt]
        Rejection regions : \(\chi^2 \geq \chi^2_{0.05, 5}=11.070\)\\*[5pt]
        Rejection decision : Don't need to reject null hypothesis\\*[5pt]
        Conclusion : Screen duration is normally distributed with $\mu=61.0587$ and $\sigma=10.3099$\\*[5pt]
    \item \textbf{Data set : Average Notification}\\*[10pt]
        Type of distribution : Normal distribution\\*[5pt]
        Known parameter : 0\\*[5pt]
        Unknown parameter ($m$) : 2, which are \(\mu, \sigma\)\\*[5pt]
        $H_0$ : Screen duration is normally distributed with $\mu=93.7671$ and $\sigma=56.1233$\\*[5pt]
        $H_a$ : Screen duration is not normally distributed with $\mu=93.7671$ and $\sigma=56.1233$\\*[5pt]
        Number of cells with the expected number of samples ($k$) : 6\\*[5pt]
        Test static \(\chi^2=\displaystyle\sum\limits^k_{i=1}\dfrac{\left(O_i-E_i\right)^2}{E_i}=1.4815\)\\*[5pt]
        Significant level \(\left(\alpha\right)\) : 0.05\\*[5pt]
        Degree of freedom 1 \((\nu_1)\) : $k - 1 - m = 6 - 1 - 2 \Rightarrow 3$\\*[5pt]
        Cutoff of non-rejection region : 7.8147\\*[5pt]
        Degree of freedom 2 \((\nu_2)\) : $k - 1 = 6 - 1 \Rightarrow 5$\\*[5pt]
        Cutoff of rejection region : 11.070\\*[5pt]
        Non-rejection regions : \(\chi^2 < \chi^2_{0.05, 3}=7.8147\)\\*[5pt]
        Rejection regions : \(\chi^2 \geq \chi^2_{0.05, 5}=11.070\)\\*[5pt]
        Rejection decision : Don't need to reject null hypothesis\\*[5pt]
        Conclusion : Screen duration is normally distributed with $\mu=93.7671$ and $\sigma=56.1233$\\*[5pt]
\end{enumerate}
\subsection*{Hypothesis Test}
\begin{enumerate}
    \item \textbf{Data set : Average screen duration}\\*[10pt]
        Test hypothesis claims that \textit{average screen duration} is greater than 7 Hrs.\\
        Collect data from 55 people. Sample mean & sample standard deviation : 6.5771 \& 2.8195 Hrs respectively.\\*[5pt]
        Case of Hypothesis test : Large sample size\\*[5pt]
        Upper-tailed, lower-tailed or two-tailed test : upper-tailed\\*[5pt]
        \textbf{7 steps test}
        \begin{enumerate}[label=1.\arabic*]
            \item Choose parameter of interest (\(\mu\))\\*[10pt]
                Parameter : \(\mu\)
            \item Specify null value (\(\mu_0\)) and null hypothesis \((H_0)\)\\
            \begin{equation}
                \begin{split}
                    \mu &: \textrm{average screen duration}\\
                    \mu_0 &: 7\\
                    H_0 &: \mu = 7
                \end{split}
            \end{equation}
            \item State alternative hypothesis \((H_a)\)
            \begin{equation}
                H_a : \mu > 7
            \end{equation}
            \item Compute test statistic \((z)\)
            \begin{equation}
                    z = \dfrac{\overline{X}-\mu_0}{\frac{S}{\sqrt{n}}} = \dfrac{6.2709-7}{\frac{2.4862}{\sqrt{46}}} \Rightarrow -1.989
            \end{equation}
            \item Indicate significance level \((\alpha)\) and find rejection region \((z_a)\)
            \begin{equation*}
                \alpha &: 0.05
            \end{equation*}
            \begin{equation}
                \begin{align}
                    1-\alpha &= 1-0.05 &&= 0.95\\
                    z_a &= z_{0.05} &&= 1.6449\\
                \end{align}
            \end{equation}
            \begin{equation*}
                z &\geq 1.6449
            \end{equation*}
            \item Determine whether we reject null hypothesis or not\\*[10pt]
                Test static does not fall inside the rejection region. Null hypothesis is not rejected.
            \item Conclude the problem\\*[10pt]
            We do not reject the null hypothesis. Average screen duration is 7 Hrs.
        \end{enumerate}
    \item \textbf{Data set : Average notification}\\*[10pt]
        Test hypothesis claims that \textit{average notification} is greater than 100 times.\\
        Collect data from 55 people. Sample mean & sample standard deviation : 103.7506 \& 74.7360 times respectively.\\*[5pt]
        Case of Hypothesis test : Large sample size\\*[5pt]
        Upper-tailed, lower-tailed or two-tailed test : upper-tailed\\*[5pt]
        \textbf{7 steps test}
        \begin{enumerate}[label=2.\arabic*]
            \item Choose parameter of interest (\(\mu\))\\*[10pt]
                Parameter : \(\mu\)
            \item Specify null value (\(\mu_0\)) and null hypothesis \((H_0)\)\\
            \begin{equation}
                \begin{split}
                    \mu &: \textrm{average notification}\\
                    \mu_0 &: 100\\
                    H_0 &: \mu = 100
                \end{split}
            \end{equation}
            \item State alternative hypothesis \((H_a)\)
            \begin{equation}
                H_a : \mu > 100
            \end{equation}
            \item Compute test statistic \((z)\)
            \begin{equation}
                    z &= \dfrac{\overline{X}-\mu_0}{\frac{S}{\sqrt{n}}} = \dfrac{93.7671-100}{\frac{56.1233}{\sqrt{46}}} \Rightarrow -0.7532
            \end{equation}
            \item Indicate significance level \((\alpha)\) and find rejection region \((z_a)\)
            \begin{equation*}
                \alpha &: 0.01
            \end{equation*}
            \begin{equation}
                \begin{align}
                    1-\alpha &= 1-0.05 &&= 0.99\\
                    z_a &= z_{0.01} &&= 2.3263\\
                \end{align}
            \end{equation}
            \begin{equation*}
                z &\geq 2.3263
            \end{equation*}
            \item Determine whether we reject null hypothesis or not\\*[10pt]
                Test static does not fall inside the rejection region. Null hypothesis is not rejected.
            \item Conclude the problem\\*[10pt]
            We do not reject the null hypothesis. Average notification is 100 times.
        \end{enumerate}
\end{enumerate}
    \clearpage
    \chapter*{Conclusion}
From the question that we asked the respondents, we want to evaluate how demographic factors (age, weight) influence screen behavior and to analyze the impact of screen duration on daily routines, including which days of the week have the highest and lowest usage.\\ \par
We found that age has a weak negative relationship with screen duration, with a correlation coefficient of -0.3340. Weight does not have any relationship with screen duration, with a correlation coefficient of -0.0137. The median of screen duration is 6.071, the median of weight is 59.95 and the median age is 21.\\ \par
The day that the most respondents have spent their screen duration is Sunday with 23.9\% and the lowest screen duration is Monday with 28.3\%. The median of sleep duration is 7.333. However, we found that screen duration and sleep duration are not correlated with the correlation coefficient of -0.1431.\\\par
The flaw of this is that the majority of people who answer the survey are around 18 & 19 years old which might cause the result to be inaccurate.

    \clearpage
    \chapter*{Appendix}
\centering
\begin{figure}[ht]
    \begin{tikzpicture}
        \pie[text=pin]{52.2/Male, 47.8/Female}
    \end{tikzpicture}
    \caption{Pie Chart of Gender}
\end{figure}
\begin{figure}[ht]
    \begin{tikzpicture}
        \pie[text=pin]{23.91/Sunday, 15.21/Tuesday, 15.21/Thursday, 13.04/Wednesday, 13.04/Friday, 13.04/Saturday, 6.52/Monday}
    \end{tikzpicture}
    \caption{Pie chart of da with the hi hest screen time}
\end{figure}
\begin{center}
    \begin{figure}[ht]
        \begin{tikzpicture}
            \pie[text=pin]{58.7/Entertainment, 34.8/Social, 4.3/Education, 2.2/Information and Reading}
        \end{tikzpicture}
        \caption{Pie chart of the most use application type}
    \end{figure}
    \begin{figure}[ht]
        \begin{tikzpicture}
            \pie[text=pin]{10.9/Sunday, 8.7/Tuesday, 19.6/Thursday, 10.9/Wednesday, 2.2/Friday, 19.6/Saturday, 28.3/Monday}
        \end{tikzpicture}
        \caption{Pie chart of day with the lowest screen time}
    \end{figure}
    \begin{figure}[ht]
        \begin{tikzpicture}
            \pie[text=pin]{69.6/None, 2.17/Cataract, 23.9/Astigmatism, 2.17/Color blindness, 2.17/Pinguecula}
        \end{tikzpicture}
        \caption{Pie chart of other eyesight problem}
    \end{figure}\\
\end{center}\\
\begin{table}[ht]
    \caption{Left eye}
    \begin{adjustbox}{width = \textwidth, center}
        \begin{tabular}{|cc|r|r|r|r|r|r|r|r|r|r|r|r|r|r|r|}
        \hline
        \multicolumn{2}{|c|}{}                                                          & \multicolumn{1}{c|}{\cellcolor[HTML]{F4CCCC}2} & \multicolumn{1}{c|}{\cellcolor[HTML]{F4CCCC}3} & \multicolumn{1}{c|}{\cellcolor[HTML]{F4CCCC}4} & \multicolumn{1}{c|}{\cellcolor[HTML]{F4CCCC}5} & \multicolumn{1}{c|}{\cellcolor[HTML]{F4CCCC}6} & \multicolumn{1}{c|}{\cellcolor[HTML]{F4CCCC}7} & \multicolumn{1}{c|}{\cellcolor[HTML]{F4CCCC}8} & \multicolumn{1}{c|}{\cellcolor[HTML]{F4CCCC}9}  & \multicolumn{1}{c|}{\cellcolor[HTML]{F4CCCC}10} & \multicolumn{1}{c|}{\cellcolor[HTML]{F4CCCC}11} & \multicolumn{1}{c|}{\cellcolor[HTML]{F4CCCC}12} & \multicolumn{1}{c|}{\cellcolor[HTML]{F4CCCC}13} & \multicolumn{1}{c|}{\cellcolor[HTML]{D9D2E9}}                                   & \multicolumn{1}{c|}{\cellcolor[HTML]{D9D2E9}}                           & \multicolumn{1}{c|}{\cellcolor[HTML]{D9D2E9}}                                                    \\
        \multicolumn{2}{|c|}{\multirow{-2}{*}{\backslashbox{$y$}{$x$}}}                 & \multicolumn{1}{c|}{\cellcolor[HTML]{FFEBEA}3} & \multicolumn{1}{c|}{\cellcolor[HTML]{FFEBEA}4} & \multicolumn{1}{c|}{\cellcolor[HTML]{FFEBEA}5} & \multicolumn{1}{c|}{\cellcolor[HTML]{FFEBEA}6} & \multicolumn{1}{c|}{\cellcolor[HTML]{FFEBEA}7} & \multicolumn{1}{c|}{\cellcolor[HTML]{FFEBEA}8} & \multicolumn{1}{c|}{\cellcolor[HTML]{FFEBEA}9} & \multicolumn{1}{c|}{\cellcolor[HTML]{FFEBEA}10} & \multicolumn{1}{c|}{\cellcolor[HTML]{FFEBEA}11} & \multicolumn{1}{c|}{\cellcolor[HTML]{FFEBEA}12} & \multicolumn{1}{c|}{\cellcolor[HTML]{FFEBEA}13} & \multicolumn{1}{c|}{\cellcolor[HTML]{FFEBEA}14} & \multicolumn{1}{c|}{\multirow{-2}{*}{\cellcolor[HTML]{D9D2E9}$\widehat{P}(Y)$}} & \multicolumn{1}{c|}{\multirow{-2}{*}{\cellcolor[HTML]{D9D2E9}midpoint}} & \multicolumn{1}{c|}{\multirow{-2}{*}{\cellcolor[HTML]{D9D2E9}$\mathrm{mid}\cdot\widehat{P}(Y)$}} \\ \hline
        \rowcolor[HTML]{FFFFFF} 
        \cellcolor[HTML]{D0E0E3}-850           & \cellcolor[HTML]{EBF1FC}-750           & 0                                              & 0                                              & 0                                              & 0                                              & 0                                              & 0                                              & \cellcolor[HTML]{C7E9D8}0.0227                 & 0                                               & 0                                               & 0                                               & 0                                               & 0                                               & \cellcolor[HTML]{D9D2E9}0.0227                                                  & \cellcolor[HTML]{D9D2E9}-800                                            & \cellcolor[HTML]{D9D2E9}-18.18181818                                                             \\ \hline
        \rowcolor[HTML]{FFFFFF} 
        \cellcolor[HTML]{D0E0E3}-750           & \cellcolor[HTML]{EBF1FC}-650           & 0                                              & 0                                              & 0                                              & 0                                              & 0                                              & 0                                              & 0                                              & 0                                               & 0                                               & 0                                               & 0                                               & 0                                               & \cellcolor[HTML]{D9D2E9}0                                                       & \cellcolor[HTML]{D9D2E9}-700                                            & \cellcolor[HTML]{D9D2E9}0                                                                        \\ \hline
        \rowcolor[HTML]{FFFFFF} 
        \cellcolor[HTML]{D0E0E3}-650           & \cellcolor[HTML]{EBF1FC}-550           & 0                                              & 0                                              & 0                                              & 0                                              & \cellcolor[HTML]{C7E9D8}0.0227                 & \cellcolor[HTML]{C7E9D8}0.0227                 & 0                                              & 0                                               & 0                                               & 0                                               & 0                                               & 0                                               & \cellcolor[HTML]{D9D2E9}0.0455                                                  & \cellcolor[HTML]{D9D2E9}-600                                            & \cellcolor[HTML]{D9D2E9}-27.27272727                                                             \\ \hline
        \rowcolor[HTML]{FFFFFF} 
        \cellcolor[HTML]{D0E0E3}-550           & \cellcolor[HTML]{EBF1FC}-450           & 0                                              & \cellcolor[HTML]{C7E9D8}0.0227                 & 0                                              & 0                                              & 0                                              & 0                                              & 0                                              & 0                                               & 0                                               & 0                                               & 0                                               & 0                                               & \cellcolor[HTML]{D9D2E9}0.0227                                                  & \cellcolor[HTML]{D9D2E9}-500                                            & \cellcolor[HTML]{D9D2E9}-11.36363636                                                             \\ \hline
        \rowcolor[HTML]{FFFFFF} 
        \cellcolor[HTML]{D0E0E3}-450           & \cellcolor[HTML]{EBF1FC}-350           & 0                                              & 0                                              & 0                                              & \cellcolor[HTML]{8FD2B1}0.0455                 & 0                                              & 0                                              & \cellcolor[HTML]{C7E9D8}0.0227                 & 0                                               & \cellcolor[HTML]{C7E9D8}0.0227                  & 0                                               & 0                                               & 0                                               & \cellcolor[HTML]{D9D2E9}0.0909                                                  & \cellcolor[HTML]{D9D2E9}-400                                            & \cellcolor[HTML]{D9D2E9}-36.36363636                                                             \\ \hline
        \cellcolor[HTML]{D0E0E3}-350           & \cellcolor[HTML]{EBF1FC}-250           & \cellcolor[HTML]{FFFFFF}0                      & \cellcolor[HTML]{C7E9D8}0.0227                 & \cellcolor[HTML]{C7E9D8}0.0227                 & \cellcolor[HTML]{FFFFFF}0                      & \cellcolor[HTML]{FFFFFF}0                      & \cellcolor[HTML]{C7E9D8}0.0227                 & \cellcolor[HTML]{FFFFFF}0                      & \cellcolor[HTML]{C7E9D8}0.0227                  & \cellcolor[HTML]{FFFFFF}0                       & \cellcolor[HTML]{FFFFFF}0                       & \cellcolor[HTML]{FFFFFF}0                       & \cellcolor[HTML]{FFFFFF}0                       & \cellcolor[HTML]{D9D2E9}0.0909                                                  & \cellcolor[HTML]{D9D2E9}-300                                            & \cellcolor[HTML]{D9D2E9}-27.27272727                                                             \\ \hline
        \cellcolor[HTML]{D0E0E3}-250           & \cellcolor[HTML]{EBF1FC}-150           & \cellcolor[HTML]{C7E9D8}0.0227                 & \cellcolor[HTML]{FFFFFF}0                      & \cellcolor[HTML]{FFFFFF}0                      & \cellcolor[HTML]{C7E9D8}0.0227                 & \cellcolor[HTML]{C7E9D8}0.0227                 & \cellcolor[HTML]{C7E9D8}0.0227                 & \cellcolor[HTML]{FFFFFF}0                      & \cellcolor[HTML]{FFFFFF}0                       & \cellcolor[HTML]{FFFFFF}0                       & \cellcolor[HTML]{FFFFFF}0                       & \cellcolor[HTML]{FFFFFF}0                       & \cellcolor[HTML]{FFFFFF}0                       & \cellcolor[HTML]{D9D2E9}0.0909                                                  & \cellcolor[HTML]{D9D2E9}-200                                            & \cellcolor[HTML]{D9D2E9}-18.18181818                                                             \\ \hline
        \cellcolor[HTML]{D0E0E3}-150           & \cellcolor[HTML]{EBF1FC}-50            & \cellcolor[HTML]{FFFFFF}0                      & \cellcolor[HTML]{C7E9D8}0.0227                 & \cellcolor[HTML]{C7E9D8}0.0227                 & \cellcolor[HTML]{C7E9D8}0.0227                 & \cellcolor[HTML]{FFFFFF}0                      & \cellcolor[HTML]{C7E9D8}0.0227                 & \cellcolor[HTML]{FFFFFF}0                      & \cellcolor[HTML]{FFFFFF}0                       & \cellcolor[HTML]{FFFFFF}0                       & \cellcolor[HTML]{FFFFFF}0                       & \cellcolor[HTML]{FFFFFF}0                       & \cellcolor[HTML]{C7E9D8}0.0227                  & \cellcolor[HTML]{D9D2E9}0.1136                                                  & \cellcolor[HTML]{D9D2E9}-100                                            & \cellcolor[HTML]{D9D2E9}-11.36363636                                                             \\ \hline
        \cellcolor[HTML]{D0E0E3}-50            & \cellcolor[HTML]{EBF1FC}50             & \cellcolor[HTML]{8FD2B1}0.0455                 & \cellcolor[HTML]{57BB8A}0.0682                 & \cellcolor[HTML]{8FD2B1}0.0455                 & \cellcolor[HTML]{57BB8A}0.0682                 & \cellcolor[HTML]{8FD2B1}0.0455                 & \cellcolor[HTML]{8FD2B1}0.0455                 & \cellcolor[HTML]{57BB8A}0.0682                 & \cellcolor[HTML]{57BB8A}0.0682                  & \cellcolor[HTML]{FFFFFF}0                       & \cellcolor[HTML]{FFFFFF}0                       & \cellcolor[HTML]{FFFFFF}0                       & \cellcolor[HTML]{FFFFFF}0                       & \cellcolor[HTML]{D9D2E9}0.4545                                                  & \cellcolor[HTML]{D9D2E9}0                                               & \cellcolor[HTML]{D9D2E9}0                                                                        \\ \hline
        \rowcolor[HTML]{FFFFFF} 
        \cellcolor[HTML]{D0E0E3}50             & \cellcolor[HTML]{EBF1FC}150            & 0                                              & 0                                              & 0                                              & 0                                              & 0                                              & 0                                              & 0                                              & 0                                               & 0                                               & 0                                               & 0                                               & 0                                               & \cellcolor[HTML]{D9D2E9}0                                                       & \cellcolor[HTML]{D9D2E9}100                                             & \cellcolor[HTML]{D9D2E9}0                                                                        \\ \hline
        \rowcolor[HTML]{FFFFFF} 
        \cellcolor[HTML]{D0E0E3}150            & \cellcolor[HTML]{EBF1FC}250            & 0                                              & 0                                              & 0                                              & \cellcolor[HTML]{C7E9D8}0.0227                 & 0                                              & 0                                              & \cellcolor[HTML]{C7E9D8}0.0227                 & 0                                               & 0                                               & 0                                               & 0                                               & 0                                               & \cellcolor[HTML]{D9D2E9}0.0455                                                  & \cellcolor[HTML]{D9D2E9}200                                             & \cellcolor[HTML]{D9D2E9}9.090909091                                                              \\ \hline
        \rowcolor[HTML]{FFFFFF} 
        \cellcolor[HTML]{D0E0E3}250            & \cellcolor[HTML]{EBF1FC}350            & 0                                              & 0                                              & \cellcolor[HTML]{C7E9D8}0.0227                 & 0                                              & 0                                              & 0                                              & 0                                              & 0                                               & 0                                               & 0                                               & 0                                               & 0                                               & \cellcolor[HTML]{D9D2E9}0.0227                                                  & \cellcolor[HTML]{D9D2E9}300                                             & \cellcolor[HTML]{D9D2E9}6.818181818                                                              \\ \hline
        \multicolumn{2}{|c|}{\cellcolor[HTML]{FFF2CC}$\widehat{P}(X)$}                  & \cellcolor[HTML]{FFF2CC}0.0682                 & \cellcolor[HTML]{FFF2CC}0.1364                 & \cellcolor[HTML]{FFF2CC}0.1136                 & \cellcolor[HTML]{FFF2CC}0.1818                 & \cellcolor[HTML]{FFF2CC}0.0909                 & \cellcolor[HTML]{FFF2CC}0.1364                 & \cellcolor[HTML]{FFF2CC}0.1364                 & \cellcolor[HTML]{FFF2CC}0.0909                  & \cellcolor[HTML]{FFF2CC}0.0227                  & \cellcolor[HTML]{FFF2CC}0                       & \cellcolor[HTML]{FFF2CC}0                       & \cellcolor[HTML]{FFF2CC}0.0227                  & \multicolumn{1}{l|}{}                                                           & \multicolumn{1}{l|}{}                                                   & \multicolumn{1}{l|}{}                                                                            \\ \hline
        \multicolumn{2}{|c|}{\cellcolor[HTML]{FFF2CC}midpoint}                          & \cellcolor[HTML]{FFF2CC}2.5                    & \cellcolor[HTML]{FFF2CC}3.5                    & \cellcolor[HTML]{FFF2CC}4.5                    & \cellcolor[HTML]{FFF2CC}5.5                    & \cellcolor[HTML]{FFF2CC}6.5                    & \cellcolor[HTML]{FFF2CC}7.5                    & \cellcolor[HTML]{FFF2CC}8.5                    & \cellcolor[HTML]{FFF2CC}9.5                     & \cellcolor[HTML]{FFF2CC}10.5                    & \cellcolor[HTML]{FFF2CC}11.5                    & \cellcolor[HTML]{FFF2CC}12.5                    & \cellcolor[HTML]{FFF2CC}13.5                    & \multicolumn{1}{l|}{}                                                           & \multicolumn{1}{l|}{\cellcolor[HTML]{F9CB9C}$\widehat{E}(X)$}           & \cellcolor[HTML]{F9CB9C}6.340909091                                                              \\ \hline
        \multicolumn{2}{|c|}{\cellcolor[HTML]{FFF2CC}$\mathrm{mid}\cdot\widehat{P}(X)$} & \cellcolor[HTML]{FFF2CC}0.1704545455           & \cellcolor[HTML]{FFF2CC}0.4772727273           & \cellcolor[HTML]{FFF2CC}0.5113636364           & \cellcolor[HTML]{FFF2CC}1                      & \cellcolor[HTML]{FFF2CC}0.5909090909           & \cellcolor[HTML]{FFF2CC}1.022727273            & \cellcolor[HTML]{FFF2CC}1.159090909            & \cellcolor[HTML]{FFF2CC}0.8636363636            & \cellcolor[HTML]{FFF2CC}0.2386363636            & \cellcolor[HTML]{FFF2CC}0                       & \cellcolor[HTML]{FFF2CC}0                       & \cellcolor[HTML]{FFF2CC}0.3068181818            & \multicolumn{1}{l|}{}                                                           & \multicolumn{1}{l|}{\cellcolor[HTML]{F9CB9C}$\widehat{E}(Y)$}           & \cellcolor[HTML]{F9CB9C}-134.0909091                                                             \\ \hline
        \end{tabular}
    \end{adjustbox}
\end{table}
\begin{table}
    \caption{Left eye correlation}
    \begin{adjustbox}{width = 0.4\textwidth, center}
        \begin{tabular}{|cll|c|}
            \hline
            \rowcolor[HTML]{FFE599} 
            \multicolumn{3}{|c|}{\cellcolor[HTML]{FFE599}Min X}                                   & Min Y      \\ \hline
            \multicolumn{3}{|c|}{2.2357}                                                          & -800       \\ \hline
            \rowcolor[HTML]{FFE599} 
            \multicolumn{3}{|c|}{\cellcolor[HTML]{FFE599}Max X}                                   & Max Y      \\ \hline
            \multicolumn{3}{|c|}{13.8500}                                                         & 250        \\ \hline
            \rowcolor[HTML]{FFE599} 
            \multicolumn{3}{|c|}{\cellcolor[HTML]{FFE599}$\overline{X}$}                                   & $\overline{Y}$      \\ \hline
            \multicolumn{3}{|c|}{6.3647}                                                          & -142.5000  \\ \hline
            \multicolumn{3}{|c|}{\cellcolor[HTML]{FFE599}$\sum(X-\overline{X})(Y-\overline{Y})}$                 & -3074.4167 \\ \hline
            \multicolumn{3}{|c|}{\cellcolor[HTML]{FFE599}$\widehat{\mathrm{cov}}_{X, Y}=\dfrac{\sum(X-\overline{X})(Y-\overline{Y})}{n-1}}$ & -69.8731   \\ \hline
            \multicolumn{3}{|c|}{\cellcolor[HTML]{FFE599}$s_X$}                                     & 2.4974     \\ \hline
            \multicolumn{3}{|c|}{\cellcolor[HTML]{FFE599}$s_Y$}                                     & 225.1834   \\ \hline
            \multicolumn{3}{|c|}{\cellcolor[HTML]{FFE599}$\rho_{X,Y}= \frac{\widehat{\mathrm{cov}}_{X, Y}}{s_X\cdot s_Y}}$        & -0.1242    \\ \hline
        \end{tabular}
    \end{adjustbox}
\end{table}\\
\begin{table}
    \begin{adjustbox}{width = \textwidth, center}
        \caption{Right eye}
        \begin{tabular}{|cc|
            >{\columncolor[HTML]{FFFFFF}}r |
            >{\columncolor[HTML]{FFFFFF}}r |
            >{\columncolor[HTML]{FFFFFF}}r |
            >{\columncolor[HTML]{FFFFFF}}r |
            >{\columncolor[HTML]{FFFFFF}}r |r|r|
            >{\columncolor[HTML]{FFFFFF}}r |
            >{\columncolor[HTML]{FFFFFF}}r |
            >{\columncolor[HTML]{FFFFFF}}r |
            >{\columncolor[HTML]{FFFFFF}}r |
            >{\columncolor[HTML]{FFFFFF}}r |r|r|r|}
            \hline
            \multicolumn{2}{|c|}{}                                                          & \multicolumn{1}{c|}{\cellcolor[HTML]{F4CCCC}2} & \multicolumn{1}{c|}{\cellcolor[HTML]{F4CCCC}3} & \multicolumn{1}{c|}{\cellcolor[HTML]{F4CCCC}4} & \multicolumn{1}{c|}{\cellcolor[HTML]{F4CCCC}5} & \multicolumn{1}{c|}{\cellcolor[HTML]{F4CCCC}6} & \multicolumn{1}{c|}{\cellcolor[HTML]{F4CCCC}7} & \multicolumn{1}{c|}{\cellcolor[HTML]{F4CCCC}8} & \multicolumn{1}{c|}{\cellcolor[HTML]{F4CCCC}9}  & \multicolumn{1}{c|}{\cellcolor[HTML]{F4CCCC}10} & \multicolumn{1}{c|}{\cellcolor[HTML]{F4CCCC}11} & \multicolumn{1}{c|}{\cellcolor[HTML]{F4CCCC}12} & \multicolumn{1}{c|}{\cellcolor[HTML]{F4CCCC}13} & \multicolumn{1}{c|}{\cellcolor[HTML]{D9D2E9}}                                   & \multicolumn{1}{c|}{\cellcolor[HTML]{D9D2E9}}                           & \multicolumn{1}{c|}{\cellcolor[HTML]{D9D2E9}}                                                    \\
            \multicolumn{2}{|c|}{\multirow{-2}{*}{\backslashbox{$y$}{$x$}}}                 & \multicolumn{1}{c|}{\cellcolor[HTML]{FFEBEA}3} & \multicolumn{1}{c|}{\cellcolor[HTML]{FFEBEA}4} & \multicolumn{1}{c|}{\cellcolor[HTML]{FFEBEA}5} & \multicolumn{1}{c|}{\cellcolor[HTML]{FFEBEA}6} & \multicolumn{1}{c|}{\cellcolor[HTML]{FFEBEA}7} & \multicolumn{1}{c|}{\cellcolor[HTML]{FFEBEA}8} & \multicolumn{1}{c|}{\cellcolor[HTML]{FFEBEA}9} & \multicolumn{1}{c|}{\cellcolor[HTML]{FFEBEA}10} & \multicolumn{1}{c|}{\cellcolor[HTML]{FFEBEA}11} & \multicolumn{1}{c|}{\cellcolor[HTML]{FFEBEA}12} & \multicolumn{1}{c|}{\cellcolor[HTML]{FFEBEA}13} & \multicolumn{1}{c|}{\cellcolor[HTML]{FFEBEA}14} & \multicolumn{1}{c|}{\multirow{-2}{*}{\cellcolor[HTML]{D9D2E9}$\widehat{P}(Y)$}} & \multicolumn{1}{c|}{\multirow{-2}{*}{\cellcolor[HTML]{D9D2E9}midpoint}} & \multicolumn{1}{c|}{\multirow{-2}{*}{\cellcolor[HTML]{D9D2E9}$\mathrm{mid}\cdot\widehat{P}(Y)$}} \\ \hline
            \cellcolor[HTML]{D0E0E3}-950           & \cellcolor[HTML]{EBF1FC}-850           & 0                                              & 0                                              & 0                                              & 0                                              & 0                                              & \cellcolor[HTML]{FFFFFF}0                      & \cellcolor[HTML]{D5EEE2}0.0233                 & 0                                               & 0                                               & 0                                               & 0                                               & 0                                               & \cellcolor[HTML]{D9D2E9}0.0233                                                  & \cellcolor[HTML]{D9D2E9}-900                                            & \cellcolor[HTML]{D9D2E9}-20.93023256                                                             \\ \hline
            \cellcolor[HTML]{D0E0E3}-850           & \cellcolor[HTML]{EBF1FC}-750           & 0                                              & 0                                              & 0                                              & 0                                              & 0                                              & \cellcolor[HTML]{FFFFFF}0                      & \cellcolor[HTML]{FFFFFF}0                      & 0                                               & 0                                               & 0                                               & 0                                               & 0                                               & \cellcolor[HTML]{D9D2E9}0                                                       & \cellcolor[HTML]{D9D2E9}-800                                            & \cellcolor[HTML]{D9D2E9}0                                                                        \\ \hline
            \cellcolor[HTML]{D0E0E3}-750           & \cellcolor[HTML]{EBF1FC}-650           & 0                                              & 0                                              & 0                                              & 0                                              & 0                                              & \cellcolor[HTML]{FFFFFF}0                      & \cellcolor[HTML]{FFFFFF}0                      & 0                                               & 0                                               & 0                                               & 0                                               & 0                                               & \cellcolor[HTML]{D9D2E9}0                                                       & \cellcolor[HTML]{D9D2E9}-700                                            & \cellcolor[HTML]{D9D2E9}0                                                                        \\ \hline
            \cellcolor[HTML]{D0E0E3}-650           & \cellcolor[HTML]{EBF1FC}-550           & 0                                              & 0                                              & 0                                              & 0                                              & \cellcolor[HTML]{D5EEE2}0.0233                 & \cellcolor[HTML]{D5EEE2}0.0233                 & \cellcolor[HTML]{FFFFFF}0                      & \cellcolor[HTML]{D5EEE2}0.0233                  & 0                                               & 0                                               & 0                                               & 0                                               & \cellcolor[HTML]{D9D2E9}0.0698                                                  & \cellcolor[HTML]{D9D2E9}-600                                            & \cellcolor[HTML]{D9D2E9}-41.86046512                                                             \\ \hline
            \cellcolor[HTML]{D0E0E3}-550           & \cellcolor[HTML]{EBF1FC}-450           & 0                                              & \cellcolor[HTML]{D5EEE2}0.0233                 & 0                                              & 0                                              & 0                                              & \cellcolor[HTML]{FFFFFF}0                      & \cellcolor[HTML]{FFFFFF}0                      & 0                                               & 0                                               & 0                                               & 0                                               & 0                                               & \cellcolor[HTML]{D9D2E9}0.0233                                                  & \cellcolor[HTML]{D9D2E9}-500                                            & \cellcolor[HTML]{D9D2E9}-11.62790698                                                             \\ \hline
            \cellcolor[HTML]{D0E0E3}-450           & \cellcolor[HTML]{EBF1FC}-350           & \cellcolor[HTML]{D5EEE2}0.0233                 & 0                                              & 0                                              & \cellcolor[HTML]{ABDDC5}0.0465                 & 0                                              & \cellcolor[HTML]{D5EEE2}0.0233                 & \cellcolor[HTML]{D5EEE2}0.0233                 & 0                                               & 0                                               & 0                                               & 0                                               & 0                                               & \cellcolor[HTML]{D9D2E9}0.1163                                                  & \cellcolor[HTML]{D9D2E9}-400                                            & \cellcolor[HTML]{D9D2E9}-46.51162791                                                             \\ \hline
            \cellcolor[HTML]{D0E0E3}-350           & \cellcolor[HTML]{EBF1FC}-250           & 0                                              & 0                                              & \cellcolor[HTML]{D5EEE2}0.0233                 & 0                                              & 0                                              & \cellcolor[HTML]{FFFFFF}0                      & \cellcolor[HTML]{FFFFFF}0                      & 0                                               & 0                                               & 0                                               & 0                                               & 0                                               & \cellcolor[HTML]{D9D2E9}0.0233                                                  & \cellcolor[HTML]{D9D2E9}-300                                            & \cellcolor[HTML]{D9D2E9}-6.976744186                                                             \\ \hline
            \cellcolor[HTML]{D0E0E3}-250           & \cellcolor[HTML]{EBF1FC}-150           & 0                                              & 0                                              & 0                                              & \cellcolor[HTML]{D5EEE2}0.0233                 & \cellcolor[HTML]{D5EEE2}0.0233                 & \cellcolor[HTML]{D5EEE2}0.0233                 & \cellcolor[HTML]{FFFFFF}0                      & 0                                               & \cellcolor[HTML]{D5EEE2}0.0233                  & 0                                               & 0                                               & 0                                               & \cellcolor[HTML]{D9D2E9}0.093                                                   & \cellcolor[HTML]{D9D2E9}-200                                            & \cellcolor[HTML]{D9D2E9}-18.60465116                                                             \\ \hline
            \cellcolor[HTML]{D0E0E3}-150           & \cellcolor[HTML]{EBF1FC}-50            & 0                                              & \cellcolor[HTML]{D5EEE2}0.0233                 & \cellcolor[HTML]{D5EEE2}0.0233                 & 0                                              & 0                                              & \cellcolor[HTML]{D5EEE2}0.0233                 & \cellcolor[HTML]{D5EEE2}0.0233                 & 0                                               & 0                                               & 0                                               & 0                                               & \cellcolor[HTML]{D5EEE2}0.0233                  & \cellcolor[HTML]{D9D2E9}0.1163                                                  & \cellcolor[HTML]{D9D2E9}-100                                            & \cellcolor[HTML]{D9D2E9}-11.62790698                                                             \\ \hline
            \cellcolor[HTML]{D0E0E3}-50            & \cellcolor[HTML]{EBF1FC}50             & \cellcolor[HTML]{ABDDC5}0.0465                 & \cellcolor[HTML]{81CCA8}0.0698                 & \cellcolor[HTML]{ABDDC5}0.0465                 & \cellcolor[HTML]{57BB8A}0.093                  & \cellcolor[HTML]{ABDDC5}0.0465                 & \cellcolor[HTML]{ABDDC5}0.0465                 & \cellcolor[HTML]{ABDDC5}0.0465                 & \cellcolor[HTML]{81CCA8}0.0698                  & 0                                               & 0                                               & 0                                               & 0                                               & \cellcolor[HTML]{D9D2E9}0.4651                                                  & \cellcolor[HTML]{D9D2E9}0                                               & \cellcolor[HTML]{D9D2E9}0                                                                        \\ \hline
            \cellcolor[HTML]{D0E0E3}50             & \cellcolor[HTML]{EBF1FC}150            & 0                                              & 0                                              & 0                                              & 0                                              & 0                                              & \cellcolor[HTML]{FFFFFF}0                      & \cellcolor[HTML]{FFFFFF}0                      & 0                                               & 0                                               & 0                                               & 0                                               & 0                                               & \cellcolor[HTML]{D9D2E9}0                                                       & \cellcolor[HTML]{D9D2E9}100                                             & \cellcolor[HTML]{D9D2E9}0                                                                        \\ \hline
            \cellcolor[HTML]{D0E0E3}150            & \cellcolor[HTML]{EBF1FC}250            & 0                                              & 0                                              & 0                                              & \cellcolor[HTML]{D5EEE2}0.0233                 & 0                                              & \cellcolor[HTML]{FFFFFF}0                      & \cellcolor[HTML]{D5EEE2}0.0233                 & 0                                               & 0                                               & 0                                               & 0                                               & 0                                               & \cellcolor[HTML]{D9D2E9}0.0465                                                  & \cellcolor[HTML]{D9D2E9}200                                             & \cellcolor[HTML]{D9D2E9}9.302325581                                                              \\ \hline
            \cellcolor[HTML]{D0E0E3}250            & \cellcolor[HTML]{EBF1FC}350            & 0                                              & 0                                              & \cellcolor[HTML]{D5EEE2}0.0233                 & 0                                              & 0                                              & \cellcolor[HTML]{FFFFFF}0                      & \cellcolor[HTML]{FFFFFF}0                      & 0                                               & 0                                               & 0                                               & 0                                               & 0                                               & \cellcolor[HTML]{D9D2E9}0.0233                                                  & \cellcolor[HTML]{D9D2E9}300                                             & \cellcolor[HTML]{D9D2E9}6.976744186                                                              \\ \hline
            \multicolumn{2}{|c|}{\cellcolor[HTML]{FFF2CC}$\widehat{P}(X)$}                  & \cellcolor[HTML]{FFF2CC}0.0698                 & \cellcolor[HTML]{FFF2CC}0.1163                 & \cellcolor[HTML]{FFF2CC}0.1163                 & \cellcolor[HTML]{FFF2CC}0.186                  & \cellcolor[HTML]{FFF2CC}0.093                  & \cellcolor[HTML]{FFF2CC}0.1395                 & \cellcolor[HTML]{FFF2CC}0.1395                 & \cellcolor[HTML]{FFF2CC}0.093                   & \cellcolor[HTML]{FFF2CC}0.0233                  & \cellcolor[HTML]{FFF2CC}0                       & \cellcolor[HTML]{FFF2CC}0                       & \cellcolor[HTML]{FFF2CC}0.0233                  & \multicolumn{1}{l|}{}                                                           & \multicolumn{1}{l|}{}                                                   & \multicolumn{1}{l|}{}                                                                            \\ \hline
            \multicolumn{2}{|c|}{\cellcolor[HTML]{FFF2CC}midpoint}                          & \cellcolor[HTML]{FFF2CC}2.5                    & \cellcolor[HTML]{FFF2CC}3.5                    & \cellcolor[HTML]{FFF2CC}4.5                    & \cellcolor[HTML]{FFF2CC}5.5                    & \cellcolor[HTML]{FFF2CC}6.5                    & \cellcolor[HTML]{FFF2CC}7.5                    & \cellcolor[HTML]{FFF2CC}8.5                    & \cellcolor[HTML]{FFF2CC}9.5                     & \cellcolor[HTML]{FFF2CC}10.5                    & \cellcolor[HTML]{FFF2CC}11.5                    & \cellcolor[HTML]{FFF2CC}12.5                    & \cellcolor[HTML]{FFF2CC}13.5                    & \multicolumn{1}{l|}{}                                                           & \multicolumn{1}{l|}{\cellcolor[HTML]{F9CB9C}$\widehat{E}(X)$}           & \cellcolor[HTML]{F9CB9C}6.406976744                                                              \\ \hline
            \multicolumn{2}{|c|}{\cellcolor[HTML]{FFF2CC}$\mathrm{mid}\cdot\widehat{P}(X)$} & \cellcolor[HTML]{FFF2CC}0.1744186047           & \cellcolor[HTML]{FFF2CC}0.4069767442           & \cellcolor[HTML]{FFF2CC}0.523255814            & \cellcolor[HTML]{FFF2CC}1.023255814            & \cellcolor[HTML]{FFF2CC}0.6046511628           & \cellcolor[HTML]{FFF2CC}1.046511628            & \cellcolor[HTML]{FFF2CC}1.186046512            & \cellcolor[HTML]{FFF2CC}0.8837209302            & \cellcolor[HTML]{FFF2CC}0.2441860465            & \cellcolor[HTML]{FFF2CC}0                       & \cellcolor[HTML]{FFF2CC}0                       & \cellcolor[HTML]{FFF2CC}0.3139534884            & \multicolumn{1}{l|}{}                                                           & \multicolumn{1}{l|}{\cellcolor[HTML]{F9CB9C}$\widehat{E}(Y)$}           & \cellcolor[HTML]{F9CB9C}-141.8604651                                                             \\ \hline
        \end{tabular}
    \end{adjustbox}
\end{table}
\begin{table}
    \caption{Right eye correlation}
    \begin{adjustbox}{width = 0.4\textwidth, center}
        \begin{tabular}{|cll|c|}
            \hline
            \rowcolor[HTML]{FFE599} 
            \multicolumn{3}{|c|}{\cellcolor[HTML]{FFE599}Min X}                                   & Min Y      \\ \hline
            \multicolumn{3}{|c|}{2.2357}                                                          & -875       \\ \hline
            \rowcolor[HTML]{FFE599} 
            \multicolumn{3}{|c|}{\cellcolor[HTML]{FFE599}Max X}                                   & Max Y      \\ \hline
            \multicolumn{3}{|c|}{13.8500}                                                         & 250        \\ \hline
            \rowcolor[HTML]{FFE599} 
            \multicolumn{3}{|c|}{\cellcolor[HTML]{FFE599}$\overline{X}$}                                   & $\overline{Y}$      \\ \hline
            \multicolumn{3}{|c|}{6.4204}                                                          & -151.7442  \\ \hline
            \multicolumn{3}{|c|}{\cellcolor[HTML]{FFE599}$\sum(X-\overline{X})(Y-\overline{Y})}$                 & -3016.2099 \\ \hline
            \multicolumn{3}{|c|}{\cellcolor[HTML]{FFE599}$\widehat{\mathrm{cov}}_{X, Y}=\dfrac{\sum(X-\overline{X})(Y-\overline{Y})}{n-1}}$ & -70.1444   \\ \hline
            \multicolumn{3}{|c|}{\cellcolor[HTML]{FFE599}$s_X$}                                     & 2.4992     \\ \hline
            \multicolumn{3}{|c|}{\cellcolor[HTML]{FFE599}$s_Y$}                                     & 244.4865   \\ \hline
            \multicolumn{3}{|c|}{\cellcolor[HTML]{FFE599}$\rho_{X,Y}= \frac{\widehat{\mathrm{cov}}_{X, Y}}{s_X\cdot s_Y}}$        & -0.1148    \\ \hline
        \end{tabular}
    \end{adjustbox}
\end{table}
    \clearpage
\end{document}